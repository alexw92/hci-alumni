%Alex
\usecase{Registrierung}
\usecasepart{Ausfüllen des Eingabeformulars}
Nachdem sich der Nutzer einen Überblick über die Seite geschaffen hat und nun über viele Funktionen, die die Seite
bietet teils gut, teils weniger gut informiert wurde, hat er sich nun für eine kostenlosen Registrierung entschieden, um das Alumni-Portal
in größerem Umfang zu nutzen.

\subsubsection*{Positive Beobachtungen}
Von der Hauptseite des Portals befindet sich der Menüpunkt "Registrieren" gut sichtbar zusammen mit einer Reihe anderer Aktionen,
wie bereits in obigem UseCase beschrieben. Das Auffinden dieses Menüpunkts stellt keine Probleme da und User kann nach einem Klick darauf direkt mit der Registrierung
beginnen.


\problem{Fehlende Erklärung des roten Sterns ”*”}
{
An vielen Eingabeformularen befindet sich über der oberen rechten Kante ein roter Stern ”*”.
}
{  (Kategorie 3).
}
{
}





\problem{Funktion neben dem Email-Eingabefeld wird nicht erklärt}
{
}
{  (Kategorie 2 oder 1).
}
{
} 

\problem{Unklare oder missverständliche Eingabefelder}
{
Grundlegende Tips zur Anmeldung fehlen...
\begin{enumerate}
	\item {“Berufliche Einbindung” → Popup neben “Branche” missverständlich}
	\item {Abschluss-Auswahl: Ist der bereits erreiche Abschluss oder der aktuelle in Arbeit Abschluss gemeint? }
	\item {Mitgliedschaft in anderen Vereinen wird nicht genau erklärt}
	
\end{enumerate}

}
{  (Kategorie 3).
}
{
} 


\problem{ Kein HTML-Link bei "Informationen zum Mentoring..." und andere(1)}
{
}
{  (Kategorie 1).
}
{
} 

\problem{Es wird auf eine nicht existente rechte Spalte mit Tips verwiesen}
{
}
{  (Kategorie 3).
}
{
} 

\problem{Man kann einzelne Buchstaben als Fach auswählen}
{
}
{  (Kategorie 3).
}
{
} 



\problem{Eingabe von Beginn und Ende des Studiums auf den Tag genau}
{
 Kein Mensch weiß wann er angefangen hat zu studieren oder aufgehört auf den Tag genau. Meldung unten drunter
zeigt zwar an, dass man schätzen soll, allerdings wird sowieso niemand den tag genau wisen deswegen kann man eig den tag direkt au dem auswahl rausstreichen.
}
{  (Kategorie 1).
}
{
} 

\problem{Überflüssige Date-Picker}
{
Die Date-Picker neben den Datumseingaben verführen zur schnellen und einfachen Eingabe des Datums. Allerdings
merkt man schnell das diese Freude unbegründet ist, denn es nicht möglich direkt ein Jahr einzugeben sondern ist gezwungen Jahr für
Jahr zurückzuscrollen was extrem nervig ist. (Bsp. Geburtsjahr)
}
{  (Kategorie 2).
}
{
}




\usecasepart{Abschicken des Eingabeformulars}

Nachdem der Benutzer die Eingabemasken mit allen notwendigen Parametern gefüllt hat, möchte er nun die Registrierung beenden und abschicken. Sobald
er die erste Hälfte der Registrierung beenden will, klickt er auf den "Weiter"-Button und sieht -falls vorhanden- seine fehlerhaften Eingaben. Hat er nach korrektem
Ausfüllen der ersten Formularseite auch die zweite ausgefüllt und mit einem erneuten Klick auf "Weiter" wieder alle Fehler berichtigt, ist die Registrierung abgeschlossen
und wird mit einer Erfolgsmeldung auf der Seite bestätigt.

\problem{Fehlermeldung bei Validierung als Popup realisiert}
{
(alert()) → hässlich und unschön”(3)
}
{  (Kategorie 3).
}
{
Formular Validierung erst beim Absenden des Formular → besser interaktiv
}

\problem{Fehlende/Falsche reguläre Ausdrücke bei Formularvalidierung}
{
	- Postleitzahl akzeptiert auch reine Strings(4)
	- Email überprüft keinen regex wie asdsa@asdsa.de oder so sondern akzeptiert alles(4)
}
{  (Kategorie 4).}
{
} 

\problem{Bereits vergebene Mail resultiert in Neuladen der Seite}
{
E-Mail bereits vergeben → neu laden der Seite, Hinweis über Formular; manche Felder sind nun wieder leer (müssen neu ausgefüllt werden)
}
{  (Kategorie 3).
}
{
}  
