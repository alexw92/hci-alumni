%Alex
\usecase{Registrierung}
\usecasepart{Teil 1}

\subsubsection*{Positive Beobachtungen}
... 

\problem{fehlt ein Hinweis zur Bedeutung des roten Sterns ”*”}
\descript{
}
\category{  (Kategorie 3).
}
\improvement{
}


\problem{Fehlende/Falsche reguläre Ausdrücke bei Formularvalidierung}
\descript{
	- Postleitzahl akzeptiert auch reine Strings(4)
	- Email überprüft keinen regex wie asdsa@asdsa.de oder so sondern akzeptiert alles(4)
}
\category{  (Kategorie 4).
}
\improvement{
}



\problem{Funktion neben dem Email-Eingabefeld wird nicht erklärt}
\descript{
}
\category{  (Kategorie 2 oder 1).
}
\improvement{
} 

\problem{Grundlegende Tips zur Anmeldung fehlen}
\descript{
\begin{enumerate}
	\item {“Berufliche Einbindung” → Popup neben “Branche” missverständlich}
	\item {Abschluss-Auswahl: Ist der bereits erreiche Abschluss oder der aktuelle in Arbeit Abschluss gemeint? }
	\item {Mitgliedschaft in anderen Vereinen wird nicht genau erklärt}
	
\end{enumerate}

}
\category{  (Kategorie 3).
}
\improvement{
} 


\problem{ Kein HTML-Link bei "Informationen zum Mentoring..." und andere(1)}
\descript{
}
\category{  (Kategorie 1).
}
\improvement{
} 

\problem{Es wird auf eine nicht existente rechte Spalte mit Tips verwiesen}
\descript{
}
\category{  (Kategorie 3).
}
\improvement{
} 

\problem{Man kann einzelne Buchstaben als Fach auswählen}
\descript{
}
\category{  (Kategorie 3).
}
\improvement{
} 



\problem{Fehlermeldung bei Validierung als Popup realisiert}
\descript{
(alert()) → hässlich und unschön”(3)
}
\category{  (Kategorie 3).
}
\improvement{
Formular Validierung erst beim Absenden des Formular → besser interaktiv
} 



\problem{Bereits vergebene Mail resultiert in Neuladen der Seite}
\descript{
E-Mail bereits vergeben → neu laden der Seite, Hinweis über Formular; manche Felder sind nun wieder leer (müssen neu ausgefüllt werden)
}
\category{  (Kategorie 3).
}
\improvement{
}


\problem{Eingabe von Beginn und Ende des Studiums auf den Tag genau}
\descript{
 Kein Mensch weiß wann er angefangen hat zu studieren oder aufgehört auf den Tag genau. Meldung unten drunter
zeigt zwar an, dass man schätzen soll, allerdings wird sowieso niemand den tag genau wisen deswegen kann man eig den tag direkt au dem auswahl rausstreichen.
}
\category{  (Kategorie 1).
}
\improvement{
} 

\problem{Überflüssige Date-Picker}
\descript{
Die Date-Picker neben den Datumseingaben verführen zur schnellen und einfachen Eingabe des Datums. Allerdings
merkt man schnell das diese Freude unbegründet ist, denn man kann kein direktes Jahr eingeben sondern muss Jahr für
Jahr zurückscrollen was extrem nervig ist. (Bsp. Geburtsjahr)
}
\category{  (Kategorie 2).
}
\improvement{
}  
