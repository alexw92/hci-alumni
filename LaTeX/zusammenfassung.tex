\section{Fazit}
Für die Evaluation wurde ein typischer Use-Case gewählt. Er beinhaltet das Besuchen der Startseite, gefolgt von einer Registrierung inklusive Freischaltung und/oder Zurücksetzen des Passworts sowie der Suche ehemaliger Kommilitonen. 
Im Rahmen dieses (beschränkten) Use-Cases haben sich bereits schwerwiegende Probleme gezeigt, die bei einer userfreundlichen Webseite nicht vorkommen sollten. 
Auch außerhalb des angesprochenen Use-Cases traten noch Probleme auf, diese sind aber nicht Thema dieser Arbeit und werden deswegen nicht aufgeführt. 

Zusammenfassend lässt sich sagen, dass die Webseite des Alumni-Portals deutliche Schwächen aufweist. Dissonanzen zwischen den Erwartungen des Nutzers und der Reaktion der Webseite waren häufige Probleme. Dies ist vor allem auf fehlende Information des Nutzers sowie auf nicht eingehaltene Webseiten-Konventionen zurückzuführen. Teilweise wurden Funktionen beworben, die nicht Bestandteil der Implementierung sind.

Abschließend hinterlässt das Alumni-Portal der Universität Würzburg aufgrund der aufzeigten Mängeln einen unprofessionellen und unseriösen Eindruck. 
Da das Portal die Universität und den Alumni-Verein repräsentiert, hat ein qualitativ schlechter Internetauftritt auch einen negativen Einfluss auf die Außenwirkung und den Ruf der Universität Würzburg.