\section{Fazit}\label{sec:fazit}
Für die Evaluation wurde ein typischer Use-Case gewählt. Er beinhaltet das Besuchen der Startseite, gefolgt von einer Registrierung inklusive Freischaltung und/oder Zurücksetzen des Passworts sowie die Suche nach ehemaligen Kommilitonen. 
Im Rahmen dieses Use-Case haben sich bereits schwerwiegende Probleme hinsichtlich User-Interface und Kommunikation in Form von Feedback mit dem Nutzer gezeigt. Das Portal bleibt dabei in mehreren Aspekten hinter dem aktuellen Standard moderner Webseiten zurück. 

Zusammenfassend lässt sich sagen, dass die Webseite des Alumni-Portals deutliche Schwächen aufweist. Dissonanzen zwischen den Erwartungen des Nutzers und der Reaktion der Webseite sind ein häufiges Problem.
Zurückführen lässt sich dies überwiegend auf die mangelnde Kommunikation von Informationen und Systemfeedback an den Nutzer.
Zum Teil werden gar Funktionen beworben, die schlichtweg nicht Bestandteil der aktuellen Implementierung der Webseite sind.

Abschließend hinterlässt das Alumni-Portal der Universität Würzburg aufgrund der aufzeigten Mängel einen unprofessionellen und unseriösen Eindruck. 
Da das Portal die Universität und den Alumni-Verein repräsentiert, hat ein qualitativ schlechter Internetauftritt auch einen negativen Einfluss auf die Außenwirkung und den Ruf der Universität Würzburg. 