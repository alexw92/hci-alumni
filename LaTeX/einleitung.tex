\section{Einleitung}
\subsection{Motivation}
Die Universität Würzburg möchte mit dem Alumni-Portal die Zusammenarbeit und den Austausch unter den Alumni der Universität fördern. 
Beim Portal handelt es sich um eine von der Universität verwaltete Webseite, die als exklusives soziales Netzwerk nur für Studenten und Absolventen der Universität fungiert. 
Das Hauptaugenmerk liegt darauf, den Alumni eine Plattform zu bieten, über die sie soziale Interaktionen tätigen können, wie ehemalige Kommilitonen/Bekannte zu finden, sowie mit diesen in Kontakt zu bleiben. Der Internetauftritt dient weiterhin zum Bewerben diverser Veranstaltungen des Alumni-Vereins.  

Diese Arbeit enthält eine Evaluation des Alumni-Portals aus der Sicht der Mensch-Computer-Interaktion. 

\subsection{Verwendete Methode}
Zum Zweck der Evaluation wird die Methode des Cognitive Walkthrough angewendet.
Die Methodik unterteilt sich in drei Phasen: Vorbereitung, Analyse sowie Zusammenfassung von Ergebnissen und Verbesserungsvorschlägen. 
Dabei wird ein typischer Use-Case aus der Sicht des Anwenders definiert, anhand dessen eine Evaluation durchgeführt wird. Alle während der Ausführung dieses Use-Case auftretenden Probleme werden im Cognitive Walkthrough berichtet, analysiert und bewertet.

Die vorliegende Arbeit ist in Anlehnung an diese drei Phasen strukturiert: Der nachfolgende Abschnitt befasst sich mit den Vorbereitungsschritten für die Evaluation. Anschließend folgt in Kapitel~\ref{sec:eval} die eigentliche Analyse des Use-Case sowie dessen Handlungssequenzen durchgeführt, welche in Kapitel~\ref{sec:fazit} noch einmal zusammengefasst wird.