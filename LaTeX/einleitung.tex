\section{Einleitung}
\subsection{Motivation}
Die Universität Würzburg möchte mit dem Alumniportal die Zusammenarbeit mit und den Austausch unter den Alumni der Universität fördern. 
Beim Alumniportal handelt es sich um eine von der Universität verwaltete Webseite, die als exklusives soziales Netzwerk nur für Studenten und Absolventen der Universtiät fugiert. 
Das Hauptaugenmerk liegt darauf, dass Alumni sich gegenseitig finden, sowie Kontakt halten können und Veranstaltungen des Alumnivereins beworben werden können. 

Diese Arbeit enthält eine Evaluation des Portals aus der Sicht der Mensch-Computer-Interaktion. 

\subsection{Verwendete Methode}
Zum Zweck der Evaluation wird die Methode des Cognitive Walkthrough angewendet.
Die Methode unterteilt sich in drei Phasen: Vorbereitung, Analyse sowie Zusammenfassung von Ergebnissen und Verbesserungsvorschlägen. 
TODO NOCH 1,2 Sätze zur Methode

Die vorliegende Arbeit ist anhand dieser drei Phasen strukturiert: Der nachfolgende Abschnitt befasst sich mit den Vorbereitungsschritten für die Evaluation, deren Ergebnisse im nachfolgenden Abschnitt aufgeführt und abschließend zusammenfassend diskutiert werden.