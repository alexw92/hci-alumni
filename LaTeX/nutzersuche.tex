%Thomas
\usecase{Besucher sucht einen ehemaligen Kommilitionen}
\usecasepart{Mit persönlichen Zugangsdaten am Portal anmelden}

\subsubsection*{Positive Beobachtungen}
Das Anmeldeformular für das Portal ist an einer zentralen und gut einsehbaren Stelle positioniert. Dadurch ist es für den Nutzer gut einsehbar und kann jederzeit seine Verwendung finden. Zwei Eingabefelder ermöglichen das Eintippen der Zugangsdaten, mit denen sich der Benutzer am Portal authentifizieren kann. Unklarheiten bezüglich der einzugebenden Daten entstehen durch die klare Struktur und Beschriftung der Elemente nicht. Der mit \emph{Anmelden} beschriftete Button führt wie erwartet den Login aus. Abschließend ist positiv anzumerken, dass nach Absenden des Formulars der Benutzer visuelles Feedback eingeblendet bekommt. Dadurch wird veranschaulicht, dass das System die Anfrage zum Login bearbeitet.

\problem{Unauffälliges Feedback bei fehlerhafter Eingabe der Zugangsdaten}
{
	Erfolgt eine fehlerhafte Eingabe von Benutzername und Passwort durchläuft das System den kompletten Vorgang der Anmeldung. Die Login-Daten werden verarbeitet und abschließend wird die Portalseite neu geladen. Befindet sich der Nutzer zum Zeitpunkt des Login nicht auf der Startseite, so wird er dorthin weitergeleitet. Die neu aufgebaute Startseite blendet dem Nutzer Feedback in Form eines kurzen Hinweistextes ein. Dieser teilt mit, dass aufgrund ungültiger Zugangsdaten die Authentifizierung nicht möglich war. Die Platzierung sowie Formatierung des Feedback ist sehr schlicht und unauffällig gewählt. Unerfahrene Anwender übersehen eine derartige Fehlermeldung unter Umständen sehr schnell und warten auf eine Reaktion von Seiten des Systems.
}
{
	Das aufgeführte Problem ist nicht so schwerwiegend, als dass es die Funktionalität der Seite in irgendeiner Form einschränkt. Durch verminderte Aufmerksamkeit des Anwenders kann es jedoch dazu führen, dass wichtiges Feedback schlichtweg nicht wahrgenommen wird. Somit ist die Art und Weise der Rückmeldung von Webseite an Benutzer nicht auffällig genug gestaltet. Aus den genannten Gründen lassen sich diese Umstände der Kategorie \glqq leichtes Problem\grqq ~zuordnen.		
}
{
	Ziel ist es, dass der Nutzer eingeblendetes Feedback, welches vom System erzeugt wird, auffällig und gut sichtbar präsentiert bekommt. Deshalb empfiehlt sich eine entsprechende Farbgestaltung: rot für Fehlermeldungen, gelb für Warnungen und grün für erfolgreich durchgeführte Aktionen. Weiterhin ist Schriftgröße angemessen zu wählen. Eine einheitliche Platzierung des Feedback ist ebenso erstrebenswert. Bei Eingabefeldern ist es beispielsweise empfehlenswert fehlerhafte Eingaben und dazugehörige Hinweistexte in unmittelbarer Nähe des betroffenen Elements zu platzieren.
}


% Aufgabenteil
% ------------
\usecasepart{Aufrufen der Alumni-Suchfunktion über das Menü}

\subsubsection*{Positive Beobachtungen}
Das horizontal angelegte Menü unterhalb der Headergrafik ist schlicht und übersichtlich gehalten. Die darin aufgelisteten Menüpunkte sind eindeutig und leicht verständlich beschrieben. Weiterhin lässt deren Benennung einen direkten Rückschluss darauf zu, welche Funktionalität der Benutzer dahinter zu erwarten hat. So ist beispielsweise direkt verständlich, dass sich hinter dem Punkt \emph{Alumni suchen} im Menü eine Suchmaske befindet.

\problem{Suchanfrage wird bereits nach erfolgreichem Laden der Seite gestartet}
{
	Nach einem Klick auf \emph{Alumni suchen} im Menü öffnet sich direkt die Suchmaske zur Eingabe von Kontaktdaten. Zu diesem Zeitpunkt ist das Suchfeld vom Anwender unberührt: es erfolgte keine Eingabe. Nach dem erfolgreichen Seitenaufbau startet unaufgefordert der Suchvorgang mit einem leeren Suchbegriff. Als Ergebnis vermeldet das System, dass keine passenden Datensätze zu der Suche vorhanden sind. Dem Benutzer werden zusätzlich Elemente zur Navigation (vorwärts und rückwärts blättern) innerhalb der Trefferliste angezeigt. Die Funktionalität der Elemente ist bei einem leeren Ergebnis gar nicht erst möglich.
}
{
	Das sofortige starten eines Suchvorgang trotz fehlender Eingabe wirkt deplatziert und ist überflüssig. Die eingeblendeten Elemente für das Blättern in der Ergebnisliste bieten dem Anwender ebenfalls kein Mehrwert. Bedienung und Funktionalität der Webseite sind aufgrund des beschriebenen Sachverhalt nicht eingeschränkt. Herbei handelt es sich um ein leichtes Problem und wird der Kategorie 2 zugeordnet.
}
{
	Das Auslösen der Suche nach dem vollständigen Aufbau der Seite entfernen. Dadurch lässt sich sowohl die Ladezeit des Inhaltes, als auch eine überflüssige Abfrage an die Datenbank einsparen. Weiterhin sollten dem Nutzer keine überflüssigen Funktionselemente angezeigt werden, für die er zum derzeitigen Zeitpunkt keine Verwendung findet.
} 

% Aufgabenteil
% ------------
\usecasepart{Auswahl der gewünschten Suchmaske im linken Menübereich}

%\subsubsection*{Positive Beobachtungen}
%... 

\problem{Linkes Seitenmenü unübersichtlich und missverständlich gestaltet}
{
	Im linken Bereich auf der Unterseite \glqq Alumni suchen\grqq ~befindet sich ein zusätzliches Menü, über welches der Nutzer die Eingabemaske für die Suche wechseln kann. Die Visualisierung der vertikal angelegten Menüelemente (Farbe und Formatierung) ist unüblich gewählt. Es ist nicht eindeutig erkennbar, welcher Menüpunkt als aktiv gesetzt ist. Dem Anwender fehlt somit das Verständnis, welche Suchmaske ihm aktuell angezeigt wird. Bekannte Mouse-Over Effekte bei Hyperlink-Elementen finden in der Menüleiste keine Verwendung. Die dort eingesetzte Farbe Schwarz in einem Navigationselement ist weiterhin von Nachteil. Für gewöhnlich gehen Nutzer bei dieser Formatierung von normalen Textelementen aus und erwarten nicht einen Link zum Klicken.
}
{
	Die ungewöhnliche Wahl der Formatierung für Hyperlinks und die fehlende Hervorhebung des aktiv gewählten Menüelements sorgt beim Endanwender für Verwirrung. Dies wiederum führt zur falschen Verwendung der im Menü angebotenen Aktionen. Aufgrund dieser Umstände lässt sich dieses Problem in die Kategorie \emph{leichtes Problem} einstufen. Die Funktionalität der Webseite ist hierdurch nicht eingeschränkt.
}
{
	Das seitliche Menü präsenter im mittleren Bereich der Suchseite als zusätzliche horizontale Navigation platzieren. Dadurch wird die Aufmerksamkeit des Nutzers angezogen und die Existenz der unterschiedlichen Eingabemasken hervorgehoben. Zudem empfiehlt sich eine die Formatierung von Hyperlink-Elementen an die Gewohnheiten erfahrener Nutzer anzupassen: unterstreichen, eine eigene Farbe für Links und einen angemessenen Mouse-Over Effekt dieser Elemente.
}

% Aufgabenteil
% ------------
\usecasepart{Eingabe eines Suchbegriffs (Vor- und Nachname, Wohnort, etc.) in die Suchleiste}

%\subsubsection*{Positive Beobachtungen}
%... 

\problem{Suchvorgang lässt sich mit leerer Eingabe ausführen}
{
	Eine Eingabe in die Suchmaske ist nicht zwingend erforderlich. Dem Anwender wird gestattet einen Suchvorgang mit leerer Eingabe zu starten. Nach einem Neuaufbau der Seite wird ein visuelles Feedback eingeblendet, das darauf hinweist, dass mindestens ein Suchbegriff notwendig ist. Trotz gescheiterter Suche wird wiederum eine leere, nicht benötigte Trefferliste mit dazugehörigen Elementen zur Navigation angezeigt.
}
{
	Das beschriebene Probleme schränkt den Anwender der Webseite hinsichtlich Bedienung und Funktionalität nicht ein. Die Tatsache, dass eine leere Suche nicht im Vorfeld bereits von Seiten des System unterbunden wird, kann beim Benutzer jedoch auf Unverständnis treffen. Insgesamt lässt sich die Problematik der Kategorie \emph{leichtes Problem} zuordnen.
}
{
	Eine denkbare Lösung ist es, den Knopf zum Starten der Suche so lange zu deaktivieren, bis der Endanwender mindestens ein Suchbegriff / -wort eingegeben hat. Danach kann der Suchvorgang wie gewohnt über den daneben liegenden Button gestartet werden. Ein alternativer Verbesserungsvorschlag ist es, dem Nutzer unmittelbar Feedback zu präsentieren und ihn auf das Fehlen eines Suchbegriffs hinzuweisen. Dadurch wird ein unnötiges Neuladen der Webseite verhindert, sowie die damit verbundenen Wartezeit während die Seite sich aktualisiert.
}

\problem{Geforderte Eingabe für das Suchfeld ist nicht eindeutig definiert bzw. vorgegeben}
{
	Bei der Suche nach einem Alumni anhand von Kontaktdaten ist die Eingabe in das vorgesehen Feld nicht eindeutig festgelegt. Intuitiv geht der Nutzer davon aus, dass er den Namen der zu suchenden Person in das Eingabefeld eingeben soll. Bei einem ausführlichen Test wird ersichtlich, dass alle personenbezogenen Daten Gegenstand einer Suchenanfrage sein können. Darunter fällt beispielhaft der Wohnort, der Straßenname, die E-Mail Adresse, sowie derzeitige Tätigkeit und Firma in der die Person arbeitet.
}
{
	Die Suche funktioniert auf den ersten Blick so wie es der Nutzer erwartet. Ein Suchbegriff wird eingegeben und daraufhin werden alle Ergebnisse eingeblendet, die eben diesen String  beinhalten. Da nicht kommuniziert wird, welche Eingabe vom Anwender erwartet wird, ist nicht eindeutig vorhersehbar, mit welchen persönlichen Daten der Alumni die Sucheingabe abgeglichen wird. Der Umstand einer fehlenden Information hinsichtlich zu tätigender Eingabe lässt dieses Problem der Katgorie \emph{leichtes Problem} zuordnen.
}
{
	Da die Webseite eine erweiterte Suchfunktionalität dem Benutzer anbietet, ist es denkbar die grundlegende Sucheingabe auf den Vor- und Nachnamen der zu suchenden Person zu beschränken. Ein erklärender Hinweis für den Benutzer hinsichtlich gültiger Eingabewerte ist notwendig. Eine Suche mit spezifischen Angaben zur Person sollte künftig über die erweiterten Suchoptionen getätigt werden.
}
\problem{Überflüssige Anzeige der Trefferanzahl während Eingabe des Suchbegriffs}
{
	Tippt der Nutzer einen Suchbegriff ein, so wird umgehend eine Abfrage an die Datenbank gesendet, welche die Anzahl an Treffern zu dieser Eingabe zurückliefert. Erkenntlich ist dies an der Lade-Animation rechts neben dem Eingabefeld, wo nach erfolgreicher Abfrage das Resultat angezeigt wird. Diese Funktionalität bietet dem Anwender zum genannten Zeitpunkt der Eingabe keinerlei wertvollen Informationsgehalt. Zumal der Suchvorgang explizit über den Knopf \emph{Suchen} gestartet werden muss bevor eine Trefferliste angezeigt wird. Das eingeblendete Ladesymbol kann den Endanwender zusätzlich verwirren, der die Eingabe unter Umständen noch nicht beendet hat. Das System vermittelt jedoch den Eindruck, als wird die Suchabfrage bereits durchgeführt.
}
{
	Das unmittelbar während der Eingabe erscheinende Ladesymbol und die noch nicht abgeschickte Suche können den Nutzer aus dem Konzept bringen. Zudem ist die Position, sowie die Art des Feedbacks zum Zeitpunkt der Eingabe für den Nutzer in dieser Art und Weise wertlos. Somit lässt sich dieses Problem der Kategorie \emph{auffälliges Problem} zuweisen.
}
{
	Zum Zeitpunkt der Eingabe kann der Nutzer sehr gut auf die Anzahl der Treffer mit dem eben eingetippten Suchbegriff verzichten. Aus diesem Grund sollte die zudem wenig aussagekräftige Information vollständig weg gelassen werden. Durchaus interessanter - und vom Informationsgehalt wertvoller - dürfte eine \emph{Autovervollständgigung} sein. Diese schlägt dem Nutzer während der Eingabe mögliche Vervollständigungen an, die aufgrund des bereits eingegebenen String und dem Datenbestand der Alumni Benutzerliste getroffen werden können.
} 

\problem{Eingabe der Suche durchsucht alle (bzw. wahllos) Datenfelder des Benutzers}
{
	Im vorherigen Verlauf des Walkthrough ist das Problem beschrieben, dass die Sucheingabe nicht eindeutig festgelegt ist. Betrachten wir im Anschluss nun die Ergebnisliste einer Suche genauer. Es fällt auf, dass Treffer aufgelistet sind, die aufgrund der Eingabe nicht relevant für den Suchenden sind. Bei dem Betrachten der Profilseite der nicht relevanten Treffer wird ersichtlich, dass alle Angaben zur Person mit dem eingegebenen Suchbegriff abgeglichen werden. So findet das Portal bei dem Suchbgriff \glqq .de\grqq ~alle Nutzer, die sich mit einer .de E-Mail Adresse angemeldet haben. Ein weiteres Beispiel: die Anschrift eines Alumni beinhaltet einen Vornamen (z.B. Anna-Straße 24).
}
{
	Bei diesem fehlerhaften und unerwarteten Funktionalität handelt es sich um ein schwerwiegendes Problem der Kategorie 4. Aus Sicht des Nutzers arbeitet die Suchfunktion nicht nach den eigenen Vorstellungen und führt die Aktion fehlerhaft aus. Intuitiv geht der Endanwender davon aus, dass mit der normalen Suchfunktion vorranging nach dem Namen eines Alumni gesucht wird. Tauchen in der Trefferliste Personen auf, deren Namen komplett von der Eingabe abweichen, so stößt dies auf Unverständnis und verwirrt den Nutzer.
}
{
	Bei der Sucheingabe muss festgelegt sein, welche Eingaben dem Nutzer gestattet sind und dadurch ersichtlich werden, welche Ergebnisse zu erwarten sind. Im vorliegenden Szenario ist es sinnvoll die normale Suche lediglich auf Namen zu beschränken und dies im Eingabefeld per \emph{Hint} vorzugeben. Für eine detaillierte Suche steht dem Suchenden zudem ein entsprechendes Formular mit erweiterten Eingabemöglichkeiten zur Verfügung. Mit Hilfe von diesem ist es ihm gestattet seine Suche weiter zu verfeinern.
} 

% Aufgabenteil
% ------------
\usecasepart{Kommilitonen über erweiterte Suchemaske finden}

\subsubsection*{Positive Beobachtungen}
Das ausgeklappte, erweiterte Suchmenü wirkt auf den Benutzer ordentlich gegliedert und auf den ersten Blick aufgeräumt. Die zu tätigenden Eingaben sind nach Obergruppen (private und geschäftl. Kontaktdaten) strukturiert und beinhalten die zugehörigen Eingabefelder.

\problem{Zu viele Eingabefelder für den Nutzer}
{
	Die Eingabefelder sind gruppiert und wirken auf den ersten Blick aufgeräumt. Bei näherem Betrachten fällt jedoch der hohe Detailgrad auf, mit dem die Suche ausgefüllt werden kann. Viele der anzugebenden Informationen stehen dem Suchenden im Regelfall nicht zur Verfügung und sind unbekannt. Durch diese Umstände wirkt die Eingabemaske unnötig überladen.
}
{
	Das beschriebene Problem lässt sich der Kategorie \emph{unauffälliges Problem} zuordnen. Die Funktionalität der Website sowie die Suchfunktion sind aufgrund der Fülle an Eingabemöglichkeiten in keinster Weise eingeschränkt. Eine derart umfangreiche Suchmaske schreckt den Benutzer oftmals vor dem Ausfüllen ab.
	
	 aber oftmals direkt ab. Die  Auf den ersten Blick schreckt die sehr umfangreiche Suchmaske lediglich den Suchenden ab, dieses entsprechend auszufüllen.
}
{
	Eine derart umfangreiche Suchmaske schreckt den Benutzer vor der Eingabe oftmals ab. Um dies zu verhindern bietet sich an, die erweiterten Suchoptionen stark zu reduzieren. Dazu können überflüssige, bzw. zu detaillierte Formularfelder bezüglich Werdegang und geschäftlichen Kontaktdaten herausgenommen werden. Es macht wenig Sinn anhand zu spezifischer Kriterien eine Suche nach ehemaligen Kommilitonen / Alumni-Mitgliedern zu starten.
}
\problem{Speichern / laden der erweiterten Suche nicht implementiert}
{
	Dem Anwender wird am Fuß der erweiterten Suchmaske die Funktionalität angeboten, das ausgefüllte Formular zu speichern und bei späteren Suchanfragen wieder zu verwenden. Das Gegenstück, eine zuvor ausgefüllte Suche laden ist ebenfalls über einen Hyperlink erreichbar. Nach Klick auf \emph{Suche speichern} wird der Anwender aufgefordert eine Bezeichnung für die ausgefüllte Suche zu vergeben. Hinter dem Knopf \emph{Speichern}, um die Eingabe zu persistieren, verbirgt sich jedoch keine Funktionalität. Das Abspeichern der Suchmaske ist beim derzeitigen Stand der Webseite nicht implementiert. Umgekehrt ist das Laden einer zuvor gespeicherten Suche nicht durchführbar, da keine vor ausgefüllten Suchmasken abgespeichert werden konnten.
}
{
	Bei der beschriebenen Situation handelt es sich um einen klaren \emph{Showstopper}. Dem Benutzer wird eine Funktionalität zur Verfügung gestellt, die allerdings nicht die erwartete Aktion ausführt. Sie bewirkt keinerlei Aktion. Aus diesem Grund handelt es sich um \emph{schwerwiegendes Problem}, das den Anwender hinsichtlich angebotener Funktionalität der Webseite einschränkt.
}
{
	Eine Möglichkeit besteht darin, das Abspeichern der Suchmaske zu implementieren. Dadurch erhält der Benutzer die zuvor beschriebene Funktionalität geboten und kann diese wie erwartet einsetzen. Alternativ kann die Funktion \emph{Suche speichern} komplett von der Webseite entfernt werden. Es ist eher ungewöhnlich, dass Speichern einer Suchmaske anzubieten. Ein sinnvoller Anwendungsfall tritt nur selten auf und die Funktionalität bietet geringen Mehrwert für den Nutzer des Systems.
}

% Aufgabenteil
% ------------
\usecasepart{Alumni anhand der Studiengangsliste finden - Eingrenzen der Liste anhand verschiedener Filterkategorien}

\subsubsection*{Positive Beobachtungen}
Durch die große Anzahl verschiedener Studiengänge wird die daraus resultierende Liste sehr lange. Mit Hilfe der bereitgestellten Filter wird dem Anwender eine hilfreiches Werkzeug an die Hand gegeben, den Umfang der Auflistung stark zu reduzieren. Dies ist ihm sowohl anhand des Anfangsbuchstaben gestattet, als auch per Jahreszahl des Studienbeginn und -ende. 

\problem{Fehlendes Feedback beim Laden der Studiengangsliste}
{
	Entscheidet sich der Nutzer einen Alumni anhand des Studiengangs zu finden, so nutzt er die Suchfunktion hinter dem Menüpunkt \emph{Studiengangs- / Jahrgangsliste} in der linken Navigation. Nach dem Klick präsentiert sich dem Anwender zunächst ein leerer Inhaltsbereich. Zu diesem Zeitpunkt meldet das System kein Feedback an den Nutzer. Es ist in dieser Situation nicht ersichtlich, ob der Ladevorgang abgeschlossen, abgebrochen oder noch in Arbeit ist. Der Benutzer erhält den Eindruck als befände er sich auf einer Unterseite ohne entsprechenden Inhalt. Erst nach längerer Wartezeit erscheint die Liste aller Studiengänge.
}
{
	Das System berichtet dem Anwender nicht den aktuellen Zustand der Aktion. Es ist nicht ersichtlich, ob in absehbarer Zeit die Aktion beendet und der erwartete Inhalt angezeigt wird. 
	Die beschriebene Thematik lässt sich als \emph{schwerwiegendes Problem} kategorisieren. Aufgrund des fehlenden Feedbacks vom System wird der Nutzer die Unterseite wieder vorzeitig verlassen. 
}
{
	Abhilfe schafft das anzeigen von Feedback für den Nutzer. Die einfachste Möglichkeit hierfür ist eine Animation in Form eines Ladebalken oder “Spinner”, die dem wartenden Benutzer angezeigt wird. Dadurch wird signalisiert, dass das System im Hintergrund noch arbeitet und in absehbarer Zeit das Ergebnis präsentiert wird.
}

%%%%%
% BIS HIERHIN DURCH!!!!

\problem{Unübersichtliche Darstellung der Studiengangsliste in einer Spalte}
{
	Die Liste, welche alle eingetragenen Studiengänge beinhaltet, erstreckt sich über eine Spalte und kann entsprechend sehr lange werden. Durch die relativ kurzen Bezeichnungen der Studiengänge wird die Breite des Inhaltbereichs nicht vollständig ausgeschöpft. Es wird vom Nutzer erwartet, dass er entsprechend weit auf der Webseite nach unten scrollt. Nur so ist es ihm gestattet, alle Einträge der Liste sehen zu können.
}
{
	Die Umstände lassen sich der Kategorie \emph{unauffälliges Problem} zuordnen. Bei der Beschreibung handelt es sich lediglich um geringe Mängel am User-Interface. Es sind keinerlei Einschränkungen der Funktionalität dadurch gegeben.
}
{
	An dieser Stelle bietet sich ein mehrspaltiges Layout für die Liste aller Studiengänge an. Hiermit wird der Inhaltsbereich besser ausgenutzt und es entsteht beim Anwender nicht der Eindruck einer nicht enden wollenden Auflistung. Die Studiengänge lassen sich somit viel schneller auffassen ohne groß auf der Webseite nach unten scrollen zu müssen.
}
\problem{Ungültige Studiengänge erscheinen in der Liste}
{
	In der Studiengangliste sind zum Teil fehlerhafte und mit Rechtschreibfehlern behaftete Studiengänge aufgelistet. Diese entstehen unter anderem bei der Registrierung durch einen neuen Benutzer am Alumni-Portal. Das Anmeldeformular gestattet bei der Registrierung die Eingabe von individuellen Studiengängen, wodurch fehlerhafte Eingaben entstehen können. Diese Problematik ist allerdings Bestandteil eines anderen Use-Case der in diesem Walkthrough ausführlich behandelt wird.
}
{
	Aufgrund des falsch zugeordneten Studiengangs im Profil eines Alumni ist es durchaus möglich, dass der Benutzer den gewünschten Alumni nicht finden kann. Somit ist das Ziel dieses Use-Case theoretisch nicht erreichbar und es handelt es hierbei um ein \emph{auffälliges Problem}.
}
{
	Derartige Fehleingaben bei der Registrierung lassen sich durch eine Validierung der Eingabedaten des entsprechenden Formulars verhindern. Dieser Lösungsvorschlag ist Bestandteil des Use-Case [XYZ - Johannes! Verlinke es!]
}

% Aufgabenteil
% ------------
\usecasepart{Durchsuchen der Trefferliste um gewünschten Alumni zu finden (inklusive Blättern in der Liste)}

\subsubsection*{Positive Beobachtungen}
Die Trefferliste ist sauber und gut strukturiert aufgebaut. Die einzelnen Resultate heben sich mit Hilfe einer unterschiedlichen Hintergrundfarbe gut voneinander ab. Zudem erhält der Endanwender direkt in der Ergebnisliste die ersten wichtigen Informationen zu den Personen, wie der aktuelle Wohnort, ein Profilbild (sofern vorhanden) und selbstverständlich den vollständigen Namen. Weiterhin positiv anzumerken sind die Blätter-Elemente mit deren Hilfe der Anwender durch die gesamte Ergebnisliste hüpfen kann. Dabei stehen ihm folgenden Funktionen zur Verfügung: vorwärts und rückwärts blättern, sowie an den Anfang bzw. das Ende der Trefferliste springen. 

\problem{Hinweis ‘Unbekannter Typ’ bei einigen Suchtreffern}
{
	In der Liste aller Treffer erscheint bei einzelnen Resultaten der Hinweistext \glqq Unbekannter Typ\grqq ~in rot und fett gedruckter Schrift. Dieser ist zurück zu führen auf eine  fehlende Eingabe im persönlichen Profil des betroffenen Alumni. Dieser Hinweistext hat für den Anwender keinerlei Informationsgehalt und trägt viel mehr zur Verwirrung bei. Der Grund für diesen auffällig gestalteten Text ist für den Suchenden nicht ersichtlich. Die Aufmerksamkeit des Nutzer wird unnötigerweise auf diesen Hinweis gezogen.
}
{
	 Die eingeblendete Profilinformation ist für den Benutzer nicht nachvollziehbar, schränkt die Funktionalität der Website allerdings nicht ein. Aus diesem Grund ist diese Auffälligkeit lediglich der Kategorie \emph{leichtes Problem} zu zu ordnen.
}
{
	Handelt es sich um eine fehlende Eingabe von Seiten des entsprechenden Alumni kann dieser Hinweistext ebenso gut ausgeblendet und nicht präsent sein. Dadurch stellt sich dem Suchenden nicht die Fragen, welche Informationen damit verbunden sind. Eine alternative Verbesserung ist, die Meldung mit mehr Informationsgehalt auszustatten, so dass der Benutzer versteht, wieso an dieser Stelle ein auffälliger Hinweistext angezeigt wird.
}
\problem{Feste Trefferanzahl pro Seite und eingeschränkte Blätterfunktion}
{
	Bei den Suchergebnissen ist die Anzahl der Treffer fest auf fünf pro Seite festgelegt. Dadurch ist der Anwender gezwungen sehr oft zu blättern, sofern sich der gesuchte Alumni mindestens in der zweiten Hälfte der Liste befindet. Dabei wird auch die eingeschränkte Funktionalität ersichtlich, zwischen den Seiten der Ergebnisliste hin und her zu springen. Es ist dem Anwender nur gestattete eine Seite vor- bzw rückwärts zu blättern, bzw. direkt an den Anfang / das Ende der Trefferliste zu springen. Die Bedienung der Blätterfunktion ist somit aus Sicht des Nutzers stark eingeschränkt und gestaltet die Bedienung ineffizient und unangenehm.
}
{
	Die Funktionalität und Handhabung der Webseite ist durch dieses Problem nicht vollständig eingeschränkt. Durch die festen Vorgaben von Seiten des System (Anzahl Treffer pro Seite, nur eine Seite vor- oder zurückblättern) werden dem Nutzer diesbezüglich keinerlei Freiheiten eingeräumt. Die beschriebene Problematik lässt sich der Kategorie \emph{leichtes Problem} zuordnen.
	}
{
	Die Anzahl der Treffer pro Seite lassen sich variabel gestalten. Der Nutzer kann per Drop-Down Menü selbst bestimmen wie viele Resultate auf einer Seite angezeigt werden sollen. Die Blätterfunktion lässt sich dahingehend verbessern, dass nicht nur ein Vorwärts- und Rückwärtsknopf angeboten wird. Das Element zum Navigieren wird hierzu komplett umgestaltet. Wie diese Navigation aussehen kann, verdeutlicht Abb. XYZ.
}