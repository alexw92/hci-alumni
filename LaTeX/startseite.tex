%Johannes
\usecasepart{Besuchen der Startseite}
Der Nutzer surft zum ersten Mal die Startseite des Alumniportals an und möchte sich grundsätzlich über das Portal und seine Funktionen informieren. 

\subsubsection*{Positive Beobachtungen}
Positiv zu bemerken ist, dass das Portal zumindest von der Farbgebung im Corperate Design der Universität Würzburg bleibt. 
Außerdem sind die Funktionalitäten sehr übersichtlich gehalten, um den User nicht von Beginn an mit Funktionen zu überfluten. 
Allgemein ist das Design sehr schlank und schlicht gehalten, um User nicht zu überfordern. 

\problem{Weltkarte ist nur als Nutzer auswählbar}

\descript{Beim ersten Blick auf die Webseite fällt sofort die ansprechende Weltkarte ins Auge, siehe Abbildung \ref{TODO:WELTKARTE}. 
Der User möchte die Karte anklicken und erwartet eine größere und interaktive Version der Weltkarte zur Verfügung zu haben.
Der Zugriff auf die Weltkarte wird einem Besucher jedoch (unverständlicherweise) nach dem Klick verwehrt und der Nutzer wird auf eine Login-Seite verlinkt. 
Dies stellt eine Dissonanz zwischen der Erwartung des Nutzers und dem Verhalten des Systems dar.
}
\category{Da die Weltkarte derart auffällig platziert ist, lockt sie die Aufmerksamkeit des Nutzers an. Dieser wird durch die Nichtverfügbarkeit abgeschreckt. 
Die fehlende Funktionalität der interaktiven Weltkarte stellt deshalb einen schweren Fehler dar, da der Nutzer bereits hier wegen fehlender Motivation abbrechen könnte (Kategorie 4).
}
\improvement{Die Weltkarte kann zweifellos auch ohne Login zugänglich und bedienbar gemacht werden. Dadurch fühlt sich der Nutzer nicht im \glqq Stöberverhalten\grqq beeinträchtigt. 
\footnote{Obwohl nicht Teil des eigentlichen Use Cases, ist die Weltkarte außerdem auch Werbung und Aushängeschild der Universität Würzburg. Sie sollte auch für nicht-Alumni und Interessenten der Universität verfügbar gemacht werden, da diese sich überhaupt nicht anmelden können.}
}
\problem{Y}
\problem{Z} 