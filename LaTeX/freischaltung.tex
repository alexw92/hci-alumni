%Armin
\newpage
\usecase{Freischaltung}
\label{subsec:freischaltung}
Der Nutzer erhält eine Mail mit dem Freischaltcode und schließt daraufhin den Registrierungsprozess ab.

\usecasepart{Empfangen und Lesen der Mail}
Nach einer Wartezeit von 2-3 Tagen nach der Registrierung erhält der Nutzer eine Mail, die einen Freischaltcode enthält (siehe Abbildung~\ref{fig:regmail}). Dieser Freischaltcode wird benötigt, um die Registrierung abzuschließen und den angelegten Account freizuschalten.

\begin{figure}
	\centering
		\includegraphics[width=\textwidth]{figures/regmail.jpg}
	\caption{Mail mit Freischaltcode}
	\label{fig:regmail}
\end{figure}

\subsubsection*{Positive Beobachtungen}
\label{subsubsec:freischaltung_mail_positiv}

Die Mail ist allgemein sprachlich sehr freundlich gehalten, der Empfänger wird gesiezt. Es ist auch klar ersichtlich, von wem die Mail stammt und welchen Zweck sie hat.

\problem{Anrede in der Mail fehlt}
{
Beim Erhalt der Mail fällt auf, dass eine Anrede fehlt. Stattdessen wird der Nutzer direkt mit seinem Nachnamen angesprochen, anstatt mit einer passenden Anrede.
}
{
Bei einer offiziellen Seite erwartet der Nutzer die Einhaltung gewisser Konventionen (Anrede, sauberes Schriftbild, klare Sätze, keine Rechtschreibfehler), welche hier nicht gegeben ist. Die fehlende Anrede vermittelt zudem einen Mangel an Seriosität, was dem Vertrauen in die Seite abträglich ist. Der Nutzer könnte sich dadurch stark verunsichert fühlen und in Zweifel geraten, dass er sich auf einer seriösen Seite befindet. Es handelt sich hierbei also um ein auffälliges Problem (Kategorie~3).

}
{
Überarbeitung des Mailtextes unter Beachtung allgemeiner sprachlicher Konventionen.
}
\label{prob:frei:mailanrede}

\problem{Mail ist nicht formatiert}
{
Neben der fehlenden Anrede fällt ebenfalls sofort die fehlende Formatierung der Mail ins Auge. Am Anfang der Mail drängt sich ein Bild, dessen Beitrag zum Inhalt der Mail sich nicht sofort erschließt, zwischen die \glqq Anrede\grqq ~und den eigentlichen Text. Außerdem beginnt der Text sehr weit links, was je nach verwendetem Programm zu Problemen bei der Lesbarkeit führt.
}
{
Die Mängel bei der Formatierung führen nicht dazu, dass der Nutzer seinen Arbeitsablauf abbrechen muss. Es dauert für ihn nur etwas länger, sich zurechtzufinden und die Mail zu lesen. Es ist daher nur ein leichtes Problem (Kategorie~2).
}
{
Hier ist eine Überarbeitung der Formatierung, insbesondere des HTML-Anteils, nötig. Des Weiteren sollte man das Bild entweder anders positionieren, oder eventuell auch ganz weglassen.
}
\label{prob:frei:mailformat}

\problem{Erstmalige Erwähnung von stud\hbox{-}mail-Adressen}
{
In der Mail wird zum ersten Mal im gesamten Registrierungsprozess erwähnt, dass Studenten keine stud\hbox{-}mail-Adressen als Kontaktadressen verwenden sollen.
}
{
Dieses Problem könnte bei Nutzern, die noch Studenten sind, für Verwirrung und zusätzlichen, unnötigen Arbeitsaufwand sorgen. Der Nutzer wird dazu gezwungen, seine gerade gewählte E\hbox{-}Mail-Adresse nach dem Login zu ändern, damit er auch in Zukunft sicher erreichbar ist. Der Benutzer kann dies als störend empfinden, da er so lange nicht sein Ziel verfolgen kann und durch die zusätzliche Arbeit abgelenkt wird. Es handelt sich also um ein störendes, auffälliges Problem (Kategorie~3).
}
{
Dieses Problem lässt sich dadurch beheben, dass bereits bei der Eingabe der E\hbox{-}Mail-Adresse bei der Registrierung deutlich darauf hingewiesen wird, dass diese Adressen nicht verwendet werden sollten. Das Eingabefeld für die E\hbox{-}Mail-Adresse kann zusätzlich noch mit einer Validierung versehen werden, um sicherzustellen, dass keine stud\hbox{-}mail-Adressen vom Besucher gewählt werden.
}
\label{prob:frei:studmail}

\problem{Link zur Freischaltung in der Mail}
{
Der Link aus der E\hbox{-}Mail, auf den der Nutzer klickt, führt zur gleichen Seite wie der Menüpunkt \emph{Freischalten} auf dem Alumni-Portal. Dort gibt es allerdings noch einen Untermenüpunkt \emph{Freischalten} mit einer anderen URL im Menü  \emph{ Registrieren}.
}
{
Das Problem wird einem normalen Nutzer nicht auffallen, da er einfach dem Link in der E\hbox{-}Mail folgt und damit umgehend auf der Seite zur Freischaltung landet (Kategorie~0).
}
{
Es ist kein Problem, das dem Nutzer auffällt oder ihn in seiner Arbeit beeinträchtigt, dennoch ist es sinnvoll, die Links zur Freischaltung zu vereinheitlichen. Der Link zu \texttt{www.alumni.uni-wuerzburg.de} kann an dieser Stelle weggelassen werden. Für den Nutzer bedeutet es einen Umweg, und er muss sich erst auf dieser Seite zurechtfinden, um dann den Weg zum Alumni-Portal zu finden.
}
\label{prob:frei:link}

\usecasepart{Ausfüllen der Webseite zur Freischaltung}
Der Nutzer sieht nun eine Webseite mit mehreren Formularfeldern, die er ausfüllen muss, um seinen Account freizuschalten.

\subsubsection*{Positive Beobachtungen}
\label{subsubsec:freischaltung_webseite_positiv}
Es werden Tipps zur Wahl von Nutzername und Passwort angezeigt. Diese Tipps können allerdings noch vereinheitlicht werden, da manche Tipps rechts neben der Webseite stehen und sich andere zusätzlich noch zwischen den Formularfeldern wiederholen.

\problem{Eingabe der E-Mail-Adresse}
{
In das oberste Feld der Seite soll der Nutzer nochmals seine E\hbox{-}Mail-Adresse eingeben.
}
{
Hierbei handelt es sich um ein auffälliges Problem (Kategorie~3), da der Nutzer seine E\hbox{-}Mail-Adresse bereits bei der Registrierung angegeben hat, und es normalerweise keinen Grund gibt, bereits bekannte Daten nochmals einzugeben. Außerdem liegt hier, durch die erneute Eingabe der E\hbox{-}Mail-Adresse, noch eine zusätzliche, vermeidbare Fehlerquelle.
}
{
Dieses Feld ist unnötig, da die E\hbox{-}Mail-Adresse bereits bei der Registrierung eingegeben wurde. Es kann also einfach weggelassen werden. Die eindeutige Identifizierung sollte über den Freischaltcode möglich sein.
}
\label{prob:frei:emaileingabe}

\problem{Button für mailto-Links}
{
Rechts neben dem Eingabefeld für die E\hbox{-}Mail-Adresse befindet sich ein kleiner Button. Erst bei genauem Hinschauen erkennt man, dass es sich um einen Briefumschlag mit einem @-Zeichen handelt. Hat der Nutzer noch keine Mailadresse eingetragen, erhält er durch den Klick ein Pop-Up mit dem Hinweistext \emph{ Bitte geben sie eine gültige E\hbox{-}Mail-Adresse an}. Ist bereits eine Mailadresse eingetragen, öffnet sich ein andere Pop-Up, in dem er eine Anwendung für mailto-Links auswählen soll.
}
{
Der Button bringt dem Nutzer keinen Mehrwert. Auch ohne ihn wird ersichtlich, dass man in das Feld seine E\hbox{-}Mail-Adresse eintragen soll. Ist bereits eine Mailadresse vom Nutzer eingetragen, hat der Button sogar eine negative Wirkung, da der Anwender dazu aufgefordert wird, eine externe Anwendung für mailto-Links zu bestimmen. Diese Links dienen normalerweise dazu, für den Nutzer das Programm seiner Wahl zu öffnen, um damit Mails an andere Personen zu verschicken. Dem Nutzer erschließt sich hier nicht der Sinn, dies auch für seine eigene E\hbox{-}Mail-Adresse zu tun, und er wird dadurch in seinem Arbeitsfluss unterbrochen. Dieses Problem erhält daher die Kategorie~2, da es für den Nutzer eventuell störend ist, aber die Funktion die Nutzung des Angebots nicht komplett verhindert.
}
{
Der Button kann weggelassen werden, da er dem Nutzer weder Mehrwert noch zusätzliche Informationen liefert und ihn im schlimmsten Fall verwirrt und behindert.
}
\label{prob:frei:buttonmailto}

\problem{Manuelle Eingabe des Freischaltcodes}
{
Der Freischaltcode aus der E\hbox{-}Mail muss manuell in ein dafür vorgesehenes Feld eingegeben werden.
}
{
Da der Freischaltcode eine zufällige Kombination aus Groß- und Kleinbuchstaben sowie Zahlen ist, besteht bei dessen Eingabe eine erhebliche Fehlerquelle, welche den Nutzer frustrieren kann. Zudem ist die manuelle Eingabe in diesem Fall ein veraltetes Procedere. Trotzdem handelt es sich hierbei lediglich um ein störendes Vorgehen der Seite, das den Benutzer aber nicht daran hindert, sich freizuschalten. Daher erhält dieses Problem die Kategorie~2.
}
{
Der Freischaltcode wird in die URL, die zur Freischaltung führt, eingebettet, damit der Nutzer den Code nicht mehr manuell eingeben muss.
}
\label{prob:frei:codeeingabe}

\problem{Wahl von Nutzername und Passwort bei der Freischaltung}
{
Nutzername und Passwort werden erst nach der Eingabe des Freischaltcodes gewählt. 
}
{
Der Sinn dieser Vorgehensweise, seinen Nutzernamen und Passwort jetzt erst zu setzen, erschließt sich dem Nutzer möglicherweise nicht. Normalerweise werden diese Daten schon bei der Registrierung festgelegt, der Vorgang entspricht also gleich zwei Mal nicht den Erwartungen des Nutzers: Zuerst erwartet der Nutzer bei der Registrierung, dass hier Nutzername und Passwort gesetzt werden, was nicht passiert, anschließend muss er die Daten an einer unerwarteten Stelle eingeben. Insgesamt erhält das Problem also die Kategorie~2.
}
{
Nutzername und Passwort können schon bei der Registrierung festgesetzt werden.
}
\label{prob:frei:nutzerundpw}

\problem{Verfügbarkeit des gewählten Nutzernamens}
{
Bei der Wahl der Nutzernamens gibt es keine sichtbare Abfrage, ob dieser noch frei ist.
}
{
Der Nutzername muss eindeutig sein, da er später zur Anmeldung auf der Webseite dient. Daher kann es den Nutzer verunsichern, wenn er nicht weiß, ob sein gewählter Name überhaupt gültig und noch nicht vergeben ist. Außerdem wird der Anwender zu einem späteren Zeitpunkt in seinem Arbeitsfluss gestört, falls der Nutzername bereits vergeben ist und er sich dann nochmals mit der Wahl eines solchen befassen muss. Somit ist das Problem störend, aber kein unüberwindbares Hindernis und erhält die Kategorie~2.
}
{
Das Formularfeld erhält eine sofortige Validierung, die sowohl überprüft, ob der Nutzername den Vorgaben entspricht, als auch eine Rückmeldung gibt, ob der dieser noch verfügbar ist.
}
\label{prob:frei:nutzerverfuegbar}

\problem{Grüner Warntext bei der Wahl des Passworts}
{
Gibt der Nutzer ein Passwort ein, das nicht den vorgegebenen Kriterien entspricht, erscheint in grüner Schrift die Fehlermeldung: \emph{Hinweis: Es konnte keine gültige Registrierung gefunden werden. Bitte überprüfen Sie Ihre Eingaben.} (siehe Abbildung~\ref{fig:greenerror}).
}
{
Dies ist ein sehr schwerwiegendes Problem (Kategorie~4). Die Farbe \glqq Grün\grqq ~ ist allgemein mit Zustimmung, Bestätigung, und bei Webseiten mit korrekten Eingaben assoziiert. Kein Nutzer wird in einem grünen Text sofort eine Warnung oder eine Fehlermeldung sehen, und die meisten User werden aufgrund der Farbgebung die Meldung nicht einmal lesen. Zusätzlich weist der Hinweistext nicht auf das Problem hin, sondern liefert nur eine allgemeingültige, schwammig formulierte Fehlermeldung.
}
{
Als erstes muss hier die Farbgebung verändert werden. Es handelt sich um eine Fehlermeldung, diese sollte also eine rote Schriftfarbe erhalten. Des Weiteren sollte die Fehlermeldung auch auffälliger gestaltet werden (größere Schrift, Rahmen, etc.). Ebenso sollte der Hinweistext angepasst werden, da er nicht eindeutig auf den Fehler hinweist. Schließlich kann auch hier, analog zum Eingabefeld für den Nutzernamen, eine Validierung eingebaut werden, die direkt bei der Eingabe ein visuelles Feedback bietet, ob das Passwort den Vorgaben entspricht.
}
\label{prob:frei:warntextgruen}

\begin{figure}
	\centering
		\includegraphics[width=\textwidth]{figures/greenerror.png}
	\caption{Ausschnitt des Formulars mit grüner Fehlermeldung}
	\label{fig:greenerror}
\end{figure}

\problem{Längenbeschränkung des Passwortfelds}
{
Das Formularfeld für die Eingabe des Passworts ist längenbeschränkt auf 17 Zeichen. Diese Beschränkung ist in den Hinweisen nicht erwähnt.
}
{

Hierbei handelt es sich klar um ein äußerst schwerwiegendes Problem (Kategorie~4). Der Nutzer findet keinen Hinweis zu dieser Beschränkung, und merkt auch beim Tippen nicht, dass er die Maximallänge erreicht hat. Der Benutzer wählt sich also ein Kennwort, das nicht dem entspricht, was in der Datenbank gespeichert wird. Das Passwortfeld beim Login hat diese Beschränkung nicht, der Nutzer tippt also sein Passwort ein, und erhält in seinen Augen trotz korrekter Eingabe die Fehlermeldung, dass sein Kennwort falsch sei. Auch wenn der Nutzer sich nun über \emph{Passwort vergessen} ~ein neues Passwort setzt, bleibt dieses Problem bestehen. Das Problem führt also dazu, dass die Webseite für den Nutzer unbenutzbar wird, bis er durch Zufall ein Passwort wählt, das weniger als 17 Zeichen enthält.
}
{
Eine Längenbeschränkung auf 17 Zeichen ist unnötig. In Zeiten von wachsendem Sicherheitsbewusstsein und der Nutzung von Passwortverwaltungssoftware wie \glqq KeePass\grqq ~ wählen viele Nutzer längere Passwörter. Sollte man an einer Längenbeschränkung festhalten, ist es sinnvoll, hier mindestens Passwörter mit einer Länge von 30 Zeichen zu erlauben. Außerdem muss die Maximallänge zwingend deutlich in den Hinweisen vermerkt werden, und die Einhaltung dieser auch direkt bei der Eingabe validiert werden.
}
\label{prob:frei:passwortlaenge}

\usecasepart{Absenden der Daten}
Nachdem der Nutzer das Formular komplett ausgefüllt hat betätigt er den Button \emph{Speichern}, um seine Daten abzusenden und auf dem Server zu speichern.

\subsubsection*{Positive Beobachtungen}
\label{subsubsec:freischaltung_absenden_positiv}
Beim Absenden der Daten erhält der User durch drehende Zahnräder visuelles Feedback, dass seine Daten gerade verarbeitet werden. Dadurch wird dem wartenden Nutzer signalisiert, dass im Hintergrund Arbeit verrichtet wird und kein unerkannter Fehler besteht.

\problem{Validierung erst beim Absenden der Daten}
{
Die Validierung der eingegebenen Daten findet erst nach dem Klick auf den Button \emph{Speichern} statt.
}
{
Durch diese Vorgehensweise wird der Nutzer in seinem Arbeitsfluss gestört, da er sich im Fehlerfall erneut mit der bereits getätigten und aus Nutzersicht abgeschlossenen Eingabe von Daten und der Wahl von Nutzernamen und/oder Passwort befassen muss (Kategorie~2).
}
{
Wie schon bei vorherigen Problemen beschrieben, lässt sich dieses Problem dadurch lösen, dass eine Validierung der Daten interaktiv direkt bei der Eingabe erfolgt.
}
\label{prob:frei:validierung}

\problem{Speichern der Daten dauert sehr lange}
{
Nach dem Klick auf \emph{Speichern} muss der Nutzer lange warten (4-5 Sekunden), bis die Daten gespeichert wurden und er sich einloggen kann. Hier ist jedoch positiv zu vermerken, dass der Nutzer in dieser Zeit Feedback erhält (siehe \ref{subsubsec:freischaltung_absenden_positiv}).
}
{
Da der Nutzer visuelles Feedback erhält und nur ein paar Sekunden warten muss, ist es kein Problem, das den Nutzer beeinträchtigt (Kategorie~0).
}
{
Hier ist es sicher möglich, Verbesserungen an der darunter liegenden Infrastruktur (Datenbank etc.) vorzunehmen, um die Verarbeitungszeit zu verkürzen.
}
\label{prob:frei:speichern}

\subsubsection*{Fazit: Freischaltung}
Bei der Freischaltung wurden zwei schwerwiegende Probleme gefunden, die eine normale Nutzung der Seite verhindern, sowie eine Reihe von Problemen, die den Nutzer in seinem Arbeitsablauf stören. Zusammenfassend kann man sagen, dass durch die Änderungsvorschläge der bisherige Freischaltungsprozess nicht mehr nötig ist, da er in Teilen in die Registrierung vorverlegt wurde und der Rest durch vorgenommene Modifikationen nicht mehr in der ursprünglichen Form nötig ist.
