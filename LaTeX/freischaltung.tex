%Armin
\usecasepart{Freischaltung}
\label{subsec:freischaltung}

\subsubsection*{Positive Beobachtungen}
\label{subsubsec:freischaltungpositiv}
Mail ist sehr freundlich gehalten\newline
Beim Absenden der Daten erhält der Nutzer durch drehende Zahnräder visuelles Feedback, dass seine Daten gerade verarbeitet werden.

%Probleme in der Mail
\problem{Anrede in der Mail fehlt}

\descript{
Beim Erhalt der Mail fällt sofort auf, dass eine Anrede fehlt. Stattdessen steht als \glqq Ersatz\grqq für die Anrede nur der Nachname da.
}
\category{
Bei einer offiziellen Seite erwartet der Nutzer die Einhaltung gewisser Konventionen, welche hier nicht gegeben ist. Die fehlende Anrede vermittelt zudem einen Mangel an Seriosität, was dem Vertrauen in die Seite abträglich ist. Der Nutzer könnte sich dadurch stark verunsichert fühlen und in Zweifel geraten, dass er sich auf einer seriösen Seite befindet. Es handelt sich hierbei also um ein auffälliges Problem (Kategorie~3).
}
\improvement{
Überarbeitung des Mailtextes unter Beachtung allgemeiner sprachlicher Konventionen.
}

\problem{Mail ist nicht formatiert}

\descript{
Neben der fehlenden Anrede fällt auch sofort die fehlende Formatierung der Mail ins Auge. Am Anfang der Mail drängt sich ein Bild, dessen Beitrag zum Inhalt der Mail sich nicht sofort erschließt, zwischen die \glqq Anrede\grqq ~und den eigentlichen Text. Außerdem beginnt der Text sehr weit links, was je nach verwendetem Programm zu Problemen bei der Lesbarkeit führt.
}
\category{
Die Mängel bei der Formatierung führen nicht dazu, dass der Nutzer seinen Arbeitsablauf abbrechen muss. Es dauert nur etwas länger, sich zurechtzufinden und die Mail zu lesen. Es ist daher nur ein leichtes Problem (Kategorie~2).
}
\improvement{
Hier ist eine Überarbeitung der Formatierung, insbesondere des HTML-Anteils, nötig. Des Weiteren sollte man das Bild entweder anders positionieren, oder eventuell auch ganz weglassen.
}

\problem{Erstmalige Erwähnung von stud\hbox{-}mail-Adressen}

\descript{
In der Mail wird zum ersten Mal im gesamten Registrierungsprozess erwähnt, dass Studenten keine stud\hbox{-}mail-Adressen als Kontaktadressen verwenden sollen.
}
\category{
Dieses Problem könnte bei Nutzern, die noch Studenten sind, für Verwirrung und zusätzlichen, unnötigen Arbeitsaufwand sorgen. Der Nutzer wird dazu gezwungen, seine gerade gewählte E\hbox{-}Mail-Adresse nach dem Login zu ändern, damit er auch in Zukunft sicher erreichbar ist. Der Nutzer kann dies als behindernd empfinden, da er so lange nicht sein Ziel verfolgen kann und durch die zusätzliche Arbeit abgelenkt wird. Es handelt sich also um ein störendes, auffälliges Problem (Kategorie~3).
}
\improvement{
Dieses Problem lässt sich dadurch beheben, dass bereits bei der Eingabe der E\hbox{-}Mail-Adresse zur Registrierung deutlich darauf hingewiesen wird, dass diese Adressen nicht verwendet werden sollen. Das Eingabefeld für die E\hbox{-}Mail-Adresse kann zusätzlich noch mit einer Validierung versehen werden, um sicherzustellen, dass keine stud\hbox{-}mail-Adressen verwendet werden.
}

\problem{Link zur Freischaltung in der Mail}

\descript{
Der Link führt zur gleichen Seite wie der Menüpunkt \glqq Freischalten\grqq ~im Alumni-Portal. Dort gibt es allerdings noch einen Untermenüpunkt \glqq Freischalten\grqq ~mit einer anderen URL im Menü  \glqq Registrieren\grqq.
}
\category{
Das Problem wird einem normalen Nutzer nicht auffallen, da er einfach dem Link in der E\hbox{-}Mail folgt und damit auf der Freischaltungsseite landet (Kategorie~0).
}
\improvement{
Es ist kein Problem, das dem Nutzer auffällt oder ihn in seiner Arbeit beeinträchtigt, dennoch ist es sinnvoll, die Links zur Freischaltung zu vereinheitlichen. Der Link zu \texttt{www.alumni.uni-wuerzburg.de} kann an dieser Stelle weggelassen werden, da es für den Nutzer einen Umweg bedeutet, und er sich erst auf der Seite zurechtfinden muss, um dann den Weg zum Alumni-Portal zu finden.
}

%Probleme auf der Webseite zur Freischaltung
\problem{Button für mailto-Links}

\descript{

}
\category{
text  (Kategorie~2).
}
\improvement{

}

\problem{Eingabe der E-Mail-Adresse}

\descript{

}
\category{
text  (Kategorie~3).
}
\improvement{

}

\problem{Händische Eingabe des Freischaltcodes}

\descript{

}
\category{
text  (Kategorie~2).
}
\improvement{

}

\problem{Wahl von Nutzername und Passwort bei der Freischaltung}

\descript{

}
\category{
text  (Kategorie~2).
}
\improvement{

}

\problem{Verfügbarkeit des gewählten Nutzernamens}

\descript{

}
\category{
text  (Kategorie~2).
}
\improvement{

}

\problem{Grüner Warntext bei der Wahl des Passworts}

\descript{

}
\category{
text  (Kategorie~4).
}
\improvement{

}

\problem{Längenbeschränkung des Passwortfelds}

\descript{

}
\category{
text  (Kategorie~4).
}
\improvement{

}

\problem{Validierung erst beim Absenden der Daten}

\descript{

}
\category{
text  (Kategorie~2).
}
\improvement{

}

\problem{Speichern der Daten dauert sehr lange}

\descript{
Man muss nach dem Klick auf Speichern lange warten (4-5 Sekunden), bis die Daten gespeichert wurden und man sich einloggen kann. Hier ist jedoch positiv zu vermerken, dass der Nutzer in dieser Zeit Feedback erhält (siehe \ref{subsubsec:freischaltungpositiv}).
}
\category{
Da der Nutzer Feedback erhält und nur ein paar Sekunden warten muss, ist es kein Problem (Kategorie~0).
}
\improvement{
Hier ist es sicher möglich, Verbesserungen an der darunter liegenden Infrastruktur (Datenbank etc.) vorzunehmen, um die Verarbeitungszeit zu verkürzen.
}