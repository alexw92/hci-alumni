\documentclass[fontsize=12pt,a4paper]{scrartcl}

\usepackage[utf8x]{inputenc}
\usepackage[USenglish,american]{babel}
\usepackage{hyperref}
\usepackage{graphicx}
\usepackage{caption}
\usepackage{subcaption}
\usepackage{enumitem}
\usepackage{tabularx}

\bibliographystyle{apalike}

\usepackage{wrapfig}
\usepackage{breakcites}
\usepackage[autostyle=true,german=quotes]{csquotes}

%counts also paragraphs
\setcounter{secnumdepth}{4}
\newcommand{\usecase}[1]{\subsection{Use-Case: #1}}
\newcommand{\usecasepart}[1]{\subsubsection{Aufgabenteil: #1}}
\newcommand{\problem}[1]{\paragraph{Problem: #1}}
\newcommand{\descript}[1]{\subparagraph{Beschreibung:} #1} 
\newcommand{\category}[1]{\subparagraph{Kategorisierung des Problems:} #1}
\newcommand{\improvement}[1]{\subparagraph{Verbesserungsvorschlag:} #1}

\begin{document}
\titlehead{
\includegraphics[scale=1.015]{figures/uni-wb-hci-header}
}

\subject{Wintersemester 2014/2015\\ Einführung in die Mensch-Computer-Interaktion}

\title{Report}

\subtitle{Cognitive Walkthrough}

\publishers{
Benedikt Pfaff, Matrikelnummer 2060170\\
Armin Beutel, Matrikelnummer 1790705\\
Johannes Grohmann, Matrikelnummer 1808010\\
Thomas Handwerker, Matrikelnummer 1995289\\
Alexander Werthmann, Matrikelnummer 1234567\\[3em] 
\normalsize Supervisors: Chris Zimmerer, Kristof Korwisi}

\maketitle

\setcounter{page}{0}
\thispagestyle{empty}

\newpage

\pagenumbering{Roman}
\setcounter{page}{1}


\tableofcontents

\newpage

\setcounter{page}{1}
\pagenumbering{arabic}

\section{Einleitung}
\subsection{Motivation}
Die Universität Würzburg möchte mit dem Alumniportal die Zusammenarbeit mit und den Austausch unter den Alumni der Universität fördern. 
Beim Alumniportal handelt es sich um eine von der Universität verwaltete Webseite, die als exklusives soziales Netzwerk nur für Studenten und Absolventen der Universtiät fugiert. 
Das Hauptaugenmerk liegt darauf, dass Alumni sich gegenseitig finden, sowie Kontakt halten können und Veranstaltungen des Alumnivereins beworben werden können. 

Diese Arbeit enthält eine Evaluation des Portals aus der Sicht der Mensch-Computer-Interaktion. 

\subsection{Verwendete Methode}
Zum Zweck der Evaluation wird die Methode des Cognitive Walkthrough angewendet.
Die Methode unterteilt sich in drei Phasen: Vorbereitung, Analyse sowie Zusammenfassung von Ergebnissen und Verbesserungsvorschlägen. 
TODO NOCH 1,2 Sätze zur Methode

Die vorliegende Arbeit ist anhand dieser drei Phasen strukturiert: Der nachfolgende Abschnitt befasst sich mit den Vorbereitungsschritten für die Evaluation, deren Ergebnisse im nachfolgenden Abschnitt aufgeführt und abschließend zusammenfassend diskutiert werden.

\section{Vorbereitung}
\subsection{Potentielle User}
Der Kreis der potentiellen User umfasst alle Studierenden und Absolventen der Uni Würzburg, sowie Professoren, Dozenten und Mitarbeiter der Universität. Außerdem gehören dazu, ausländische (nicht deutschsprachige) und inländische Gaststudenten sowie Interessierte. Grundsätzlich wird davon ausgegangen, dass ein potentieller User Akademiker ist und/oder einen höheren Bildungsabschluss hat. Desweiteren wird unterstellt, dass der User bereits Erfahrung im Umgang mit dem Browser, dem Abrufen und Schreiben von e-Mails und dem Registrieren auf Webseiten hat. 

%hier allgemeine Aufgabenbeschreibung, was machen wir insgesamt
\subsection{Aufgabenbeschreibung}
Für die Evaluation wird das Registrieren auf der Webseite der Alumni Uni Würzburg gewählt. Dabei sollen alle Schritte bis zur erfolgreichen Registrierung durchgeführt werden. Zudem soll ein Anmeldeversuch nach erfolgreicher Registrierung durchgeführt werden.
Die Registrierung ist in folgende Schritte untergliedert:
\begin{enumerate}
	\item Aufrufen der Alumni Webseite
	\item Zurechfinden auf der Webseite und Finden des Registrierungsformulars
	\item Ausfülledes ersten Teiles des Registrierungsformulars
	\item Absenden des ersten Formulars
	\item Ausfüllen des zweiten Teiles des Registrierungsfomulares
	\item Absenden des zweiten Formulares
	\item weitere Schritte...
\end{enumerate}

In den folgenden Kapiteln werden die einzelnen Use Cases genauer definiert und analysiert, bla blubb und so.

\subsection{Mögliche Aktionen und Effekte}
\begin{enumerate}
	\item 
	\begin{itemize}
		\item Auswahl !Registrieren! in der linken Navigations-Leiste 
	\end{itemize}
	\item 
	\begin{itemize}
		\item Eingabe Persönlicher Daten
		\item Eingabe der Adresse
		\item Eingabe der Heimatadresse
		\item Eingabe Berufliche Einbindung
		\item Auswahl Alumni Angebot
		\item Auswahl zur kostenpflichtigen Mitgliedschaft
		\item Auswahl !Wie haben Sie von der Alumni Würzburg erfahren?!
		\item Markierung zur Einverständniserklärung der Datenschutzerklärung
		\item Markierung zur Einverständniserklärung der Nutzung Personenbezogener Daten
	\end{itemize}
	\item 
	\begin{itemize}
		\item Bestätigung durch den Button !Weiter!
		\item Abbruch durch den Button !Abbruch! 
	\end{itemize}
	\item 
	\begin{itemize}
		\item Eigabe der Daten des Zweiten Formulars...
	\end{itemize}
	\item 
	\begin{itemize}
		\item Absenden des zweiten Formulars...
	\end{itemize}
	\item 
	\begin{itemize}
		\item weitere Schritte...
	\end{itemize}
\end{enumerate}

\section{Evaluationsergebnisse}
Die im Rahmen des Cognitive Walkthrough gemachten positiven Beobachtungen und detektierten Probleme bezüglich Design und Bedienbarkeit werden in den Abschnitten aufgelistet. Dabei werden pro Problem jeweils die genaue Problembeschreibung, eine Kategorisierung des Problems sowie ein Verbesserungsvorschlag genannt.

Tabelle \ref{tbl:categories} listet die in diesem Zusammenhang verwendeten Kategorien auf. Die Unterteilung der gefundenen Probleme erfolgt anhand der Aufgabenbereiche und in der Reihenfolge in denen sie in unserem Use Case aufgetreten sind.

\begin{table}[h]
	\centering\begin{tabular}{|c|l|}
		\hline
		\textbf{Kategorie} & \textbf{Bedeutung} \\
		\hline
		0 & kein Problem \\
		1 & unauffälliges Problem \\
		2 & leichtes Problem \\
		3 & auffälliges Problem \\
		4 & schwerwiegendes Problem \\
		\hline
	\end{tabular}
	\caption{Problemschweregrade\label{tbl:categories}}
\end{table}
%Johannes
\newpage
\usecase{Besuchen der Startseite}
Der Nutzer öffnet zum ersten Mal die Startseite des Alumni-Portals und möchte sich grundsätzlich über das Portal und seine Funktionen informieren. Dadurch will er erfahren, ob eine Registrierung für ihn überhaupt Sinn ergibt. 

\usecasepart{Betrachten der Startseite}
Nach Eingeben der URL baut sich die Startseite auf (siehe Abbildung~\ref{fig:start}). Der Nutzer verschafft sich einen Überblick über die verfügbaren Funktionen und versucht einige von ihnen auszuprobieren. Er versucht zum Beispiel die Weltkarte anzuklicken oder wirft einen kurzer Blick auf die Stellenangebote. 

\begin{figure}[h]
	\centering
		\includegraphics[width=\textwidth]{figures/startseite.png}
	\caption{Die Startseite des Alumni-Portals}
	\label{fig:start}
\end{figure}


\subsubsection*{Positive Beobachtungen}
Positiv zu bemerken ist, dass das Portal sich in puncto Farbgebung an das Corperate Design der Universität Würzburg anpasst. 
Auch das Titelbild, sowie die beiden Logos der Universität und des Alumni-Vereins sind passend eingebunden.
Weiterhin sind die Funktionalitäten sehr übersichtlich gehalten, um den Nutzer nicht von Beginn an mit Funktionen zu überfluten. 
Allgemein ist das Design sehr schlank und schlicht gehalten, um Nutzer nicht zu überfordern und abzuschrecken.

Der Rumpf der Seite ist schmal gehalten, um auch auf Geräten mit einer weniger hohen Auflösung korrekt angezeigt werden zu können. Der Rand ist dabei in neutralem Grau gehalten und wirkt angenehm passiv.

\problem{Unstrukturiertes Erscheinungsbild}
{Trotz des recht schlanken Designs, wirkt die Webseite auf den ersten Blick recht unstrukturiert. Dies äußert sich in mehreren Punkten, die hier zusammengefasst werden. 
So werden insgesamt (exklusive Logos) 8 verschiedene Schrifttypen verwendet, die zudem nicht immer einen semantischen Unterschied signalisieren. In den beiden Textspalten finden sich beispielsweise zwei unterschiedliche Schrifttypen. 

Hinzu kommt ein viel zu kleiner Hinweistext in grau am Beginn der Seite, der darauf hinweist, dass nicht alle Funktionen ohne Login verfügbar sind.
In der linken Textspalte, ist zudem der Textsatz nicht überprüft worden, was dazu führt, dass das Wort \glqq sind\grqq~in die zweiten Zeile direkt neben die Weltkarte gesetzt wird, was auf den Nutzer sehr gequetscht wirkt.
Zusätzlich ist die Überschrift der rechten Spalte \emph{Alle Detailansichten im eingeloggten Bereich} unklar, unpassend und unverständlich und sorgt damit für Verwirrung.
}
{Durch das uneinheitliche Erscheinungsbild, wird es dem Nutzer erschwert sich auf der Seite zurecht zu finden. Dennoch sind nach längerem Suchen die Struktur und alle Funktionen der Seite erkennbar, das Problem ist also nicht besonders schwerwiegend, sondern nur geringfügig störend (Kategorie~1).
}
{Es wird eine Reduktion auf 3 bis 4 verschiedene Schrifttypen und weniger, aber dafür aussagekräftige Überschriften empfohlen. Zusätzlich mit einer deutlicheren Hervorhebung von Hinweistexten sollte dies das allgemeine Erscheinungsbild und den ersten Eindruck, den der Nutzer von der Startseite gewinnt, deutlich verbessern.
}\label{prob:start:erschbild}

\problem{Weltkarte ist nicht für Besucher verfügbar}
{Bei einem Blick auf die Startseite fällt sofort die ansprechende Weltkarte auf, siehe Abbildung \ref{fig:start}. 
Der Nutzer möchte die Karte anklicken und erwartet eine größere und interaktive Version der Weltkarte zur Verfügung zu haben.
Der Zugriff auf die Weltkarte wird einem Besucher jedoch (unverständlicherweise) nach dem Klick verwehrt und der Nutzer wird auf eine Login-Seite verlinkt. 
Dies stellt eine Dissonanz zwischen der Erwartung des Nutzers und dem Verhalten des Systems dar.
}{Da die Weltkarte derart auffällig platziert ist, lockt sie die Aufmerksamkeit des Nutzers an. Dieser wird durch die Nichtverfügbarkeit abgeschreckt. 
Die fehlende Funktionalität der interaktiven Weltkarte stellt deshalb einen schweren Fehler dar, da der Nutzer bereits hier wegen fehlender Motivation abbrechen könnte (Kategorie~4).}
{Die Weltkarte kann zweifellos auch ohne Login zugänglich und bedienbar gemacht werden.\footnote{Obwohl nicht Teil des eigentlichen Use-Cases, ist die Weltkarte außerdem auch Werbung und Aushängeschild der Universität Würzburg. Sie sollte auch für Nicht-Alumni und Interessenten der Universität verfügbar gemacht werden, da diese sich überhaupt nicht anmelden können.}
Dadurch fühlt sich der Nutzer nicht im \glqq Stöberverhalten\grqq~ beeinträchtigt.
}\label{prob:start:weltkarte}

\problem{Zweispaltiges Design}
{Der Text der Startseite ist in zwei Spalten aufgeteilt. Dies führt zu einer schlechteren Lesbarkeit, da zudem keine Hierarchie erkennbar ist, also beiden Spalten gleich viel Platz eingeräumt wird.
Dadurch wird kein Leitartikel erkennbar, was es für den Nutzer schwer macht in die Webseite einzusteigen. 
Durch die Seitenränder und das Seitenmenü findet eine weitere Einengung statt, sodass der Text stark in die Länge gezogen wird, der Nutzer somit weit nach unten scrollen muss.
Hinzu kommt, dass die linke Spalte deutlich weniger Text beinhaltet als die rechte Spalte, was dazu führt, das am Ende der Seite nur noch etwa ein Viertel der zur Verfügung stehenden Breite genutzt wird. Dies führt wiederum zu einer weiteren Streckung des Textes. 
}{Die Zweiteilung wirkt sehr ungeschickt und verhindert zudem, dass der gesamte Platz der Webseite genutzt werden kann. Es führt allerdings nicht zu einer Behinderung des Nutzers, weshalb es als leichtes Problem eingestuft werden kann (Kategorie~2).
}{Wichtig wäre hier vor allem eine klare Hierarchie vorzugeben, die sich auch in der Platzaufteilung zeigt. Dadurch wird die Aufmerksamkeit des Nutzers direkt auf diese Spalte gelenkt und es gibt einen klar erkennbaren Anfang der Seite.}\label{prob:start:design}

\problem{Scrollpanel in linker Spalte} 
{In der linken Spalte befindet sich ein Scrollpanel mit der Überschrift \emph{Aktuelles}. Dem Nutzer erscheint eben dieses jedoch vollkommen fehl am Platz.
Zunächst sind die Informationen im Contentbereich viel zu klein dargestellt und jede Nachricht ist auf 2 Zeilen begrenzt. Außerdem erübrigt sich der platzsparende Charakter eines Scrollpanels, da durch die rechte Spalte die Webseite ohnehin nach unten expandiert wird. 
Dies führt dazu, dass der Rest der linken Spalte leer bleibt, während der Nutzer im viel zu klein geratenen Panel nach unten scrollen muss.
}{
Der Sinn des gesamten Panels erschließt sich dem Nutzer keineswegs. Es spart keinen Platz, sondern führt nur dazu, dass der Nutzer gezwungen ist, selbst im Panel zu scrollen. Das führt zu Verärgerung und Verwirrung, da das Tool hier völlig falsch angewandt wurde (Kategorie~3).
}{
In diesem Fall, wäre es besser das Scrollpanel einfach wegzulassen und den Newsfeed \emph{Aktuelles} einfach als Artikel unter der Weltkarte normal anzuzeigen. Gegebenenfalls kann das Scrollpanel auch so eingestellt werden, dass es sich bis zum Seitenende expandiert, also beide Spalten ausgeglichen lang sind. 
}\label{prob:start:panel}

%\problem{Zurücksetzen der Startseite auf Englisch} 
%{TODO SOLL DAS ÜBERHAUPT MIT REIN?
%}
%{hallo
%}
%{hallo
%}
%
%\problem{Statisches Webseitendesign passt sich nicht an Auflösung an} 
%{TODO SOLL DAS ÜBERHAUPT MIT REIN?
%}
%{hallo	
%}
%{hallo
%}

\usecasepart{Betrachten der Menüs}
Im Folgenden untersucht der Nutzer die Startmenüs und ihre Funktionen etwas genauer. Zu diesem Zweck ist in Abbildung~\ref{fig:menu} das Menü noch einmal vergrößert dargestellt.

\begin{figure}[h]
	\centering
		\includegraphics[width=\textwidth]{figures/menu.png}
	\caption{Eine vergrößerte Darstellung der Menüs der Startseite}
	\label{fig:menu}
\end{figure}


\subsubsection*{Positive Beobachtungen}
Die beiden Menüs sind in Umfang und Funktion recht übersichtlich gehalten, es wurde auf die Basisfunktionen reduziert. Dadurch sind alle wichtigen Menüpunkte schnell und einfach zu erreichen. 
Außerdem ist der derzeit aktive Menüpunkt immer deutlich markiert, so dass der Nutzer zu jedem Zeitpunkt weiß, auf welcher Seite er sich im Moment befindet.

\problem{Verwendung zweier Menüs}
{Zunächst fällt dem Nutzer auf, dass sowohl ein Header-Menü, als auch ein Menü am linken Seitenrand vorhanden sind. Das ist verwirrend, da auch keine semantische Trennung in den Funktionen beider Menüs vorhanden ist. 
}{Das Vorhandensein zweier Menüs an unterschiedlicher Stelle verwirrt den Nutzer. Da dennoch alle Funktionen verfügbar sind, ist das Problem nicht besonders schwerwiegend (Kategorie~2).
}{Hier wäre eine Reduktion auf ein Menü ratsam (vorzugsweise das Header-Menü). Dadurch sind alle verfügbaren Funktionen klar sortiert und somit zugänglicher. 
}\label{prob:start:menues}

\problem{Redundante Funktionen} 
{Die Menüpunkte \emph{Start} und \emph{Alumni-Hompage}, sowie \emph{Portal in Englisch} und \emph{English version} sind jeweils redundant in beiden Menüs mit unterschiedlicher Benennung angebracht, verfolgen aber den gleichen Zweck. Außerdem ist der Menüpunkt \emph{Alumni-Portal} in beiden Menüs vorhanden.
}
{Durch die doppelte Platzierung und unterschiedliche Benennung der Menüpunkte wird dem Nutzer ein semantischer Unterschied suggeriert, der nicht vorhanden ist. Das führt zwar zur Verwirrung, die Funktionen sind aber dennoch verfügbar und auffindbar (Kategorie~1).
}
{Die redundanten Menüpunkte können entfernt werden, um dem Nutzer eine klare Übersicht über die verfügbaren Funktionen zu geben. 
}\label{prob:start:funktionen}

\problem{Menüpunkt: Logindaten vergessen} 
{Die Funktion \emph{Logindaten vergessen} ist im Menü recht unüblich. Der Nutzer nimmt an, dass die Funktionen des Menüs zur Navigation dienen. Das Problem der vergessenen Logindaten passt nicht in diesen Kontext.
}
{Der Nutzer wird durch die zusätzliche/unnötige Funktion verwirrt und von seiner eigentlichen Intention abgelenkt (Kategorie~2).
}
{Die Funktion sollte in der Nähe der Login-Funktion angebracht werden, da sie nur in diesem Kontext gebraucht wird. Außerdem ist das mittlerweile auch die übliche Platzierung, an dem ein Nutzer sie zuerst suchen würde.
}\label{prob:start:pkt:login}

\problem{Menüpunkt: Registrieren}
{Der Menüpunkt \emph{Registrieren} ist nicht im üblichen Kontext in der Nähe des Anmeldefensters angeordnet. Bei einem Klick öffnet sich ein Untermenü, das dem Nutzer zwei Optionen bietet: \emph{Registrieren} und \emph{Freischalten}.
Die Funktion \emph{Freischalten} ist vollkommen unklar. Der Nutzer weiß nicht, was die Funktion bewirkt oder wofür sie da ist. Auch der Unterschied zwischen Registrieren und Freischalten wird nicht deutlich.
}{Die Funktion ist nicht in ihrem üblichen Kontext angeordnet. Der Nutzer wird zudem durch die zusätzliche/unnötige Funktion verwirrt. Es entsteht der Eindruck, zur erfolgreichen Registrierung eine weitere Aktion ausführen zu müssen (Kategorie~3).}
{Die Registrierungsfunktion sollte in den Kontext des Anmeldefensters verschoben werden. Außerdem muss die Funktion des Freischaltens deutlicher erklärt werden. Alternativ wäre ein Zugang zum Freischalten auch nur mittels einem in der Registrierungsmail verschickten Hyperlink oder eine direkte Verlinkung nach dem ersten erfolgreichen Login denkbar.
}\label{prob:start:pkt:reg}

\problem{Menüpunkt: Freischalten} 
{Die Funktion Freischalten ist erstens bereits im Menü vorhanden, zweitens ist die Funktion dem Nutzer hier nicht klar (siehe Abschnitt~\ref{subsubsec:menuregistrieren}). 
Zusätzlich verlinken die beiden Funktionen mit dem Titel \emph{Freischalten} auf verschiedene Seiten, die aber beide die gleiche Funktion erfüllen. Für den Nutzer ist das allerdings nicht ersichtlich, da es sich um zwei verschiedene Links handelt.
}
{Durch den Mangel an Information und des Vorhandenseins zweier verschiedener Links, wird dem Nutzer zudem suggeriert es handele sich bei \emph{Freischalten} um zwei verschiedene Funktionen (Kategorie~3).
}
{
Da die Funktion bereits einmal im Menü vorhanden ist, genügt hier die Reduktion auf eine der beiden Optionen.
}\label{prob:start:pkt:frei}

\problem{Menüpunkt: Start} 
{Im linken Seitenmenü taucht zusätzlich ein weiterer Menüpunkt \emph{Start} auf. Auch hier ist die Funktion überhaupt nicht klar. Der Nutzer würde erwarten auf die Startseite der Webseite verlinkt zu werden, es geschieht jedoch eine Weiterleitung auf die Alumni-Hompage. 
}
{Dem Nutzer ist sich überhaupt nicht über die Funktion und deren Effekt im Klaren, was zu Verwirrung führt (Kategorie~3).
}
{Da der Menüpunkt ohnehin redundant mit der Funktion \emph{Alumni-Homepage} im Header-Menü ist, kann er ohne weiteres entfernt werden.
}\label{prob:start:pkt:start}

\problem{Menüpunkt: Portal in Englisch} 
{Der Menüpunkt \emph{Portal in Englisch} wirkt fehl am Platz. Sollte der Nutzer tatsächlich nach einer Übersetzung der Seite suchen, würde er dies rechts oben in der Ecke oder an sonstigen exponierten oder markanten Stellen tun. 
Der Nutzer stolpert hier im Menü nur über den Punkt, weil er ihn nicht an dieser Stelle erwartet hätte.
}
{Da der Menüpunkt recht unmissverständlich ist, hält er den Nutzer nicht lange auf (Kategorie~1).
}
{Die Funktion sollte in die rechte obere Ecke, neben anderen Sprachoptionen eingegliedert werden, wo der Nutzer sie erwarten würde.
}\label{prob:start:pkt:eng}

\problem{Menüpunkt: Impressum} 
{Die Funktion \emph{Impressum} im linken Seitenmenü wirkt ebenfalls fehl am Platz. Der Nutzer erwartet ein Impressum am Fuß der Seite, eine eigene Funktion im Menü lässt dieses schnell überladen wirken.
}
{Das Menü wird um eine unnötige Funktion erweitert, die schnell zum Überladen der Funktionen führen kann, der Nutzer findet das Impressum allerdings trotzdem falls er danach sucht (Kategorie~1).
}
{Das Impressum sollte nicht als Menüpunkt, sondern an den Fuß der Seite eingebunden werden. Dies hat sich als de-facto Standard etabliert, sodass der Nutzer es schneller findet, wenn er danach sucht.
Außerdem wird das Impressum so auch auf jeder Seite angezeigt und nicht nur auf der Startseite verlinkt. 
}\label{prob:start:pkt:imp}

\problem{Pop-Up Fenster} 
{Mehrere Aktionen der Menüs sind als Pop-Up Funktionen implementiert. Durch Klick auf \emph{Start}, \emph{Portal in Englisch}, \emph{English version} und \emph{Alumni Portal} öffnet sich ein Pop-Up Fenster, das von modernen Browsern grundsätzlich blockiert wird. 
Der Nutzer bekommt dies meistens nicht einmal mehr mit. Stattdessen erscheint nur ein kleiner Text auf der Seite, der darauf hinweist, dass die Seite in einem externen Fenster geöffnet wurde mit einem Link, um das öffnen manuell auszuführen.
Das Ergebnis ist, dass der Nutzer zweimal klicken muss, um auf die gewünschte Seite zu kommen.
}
{Da Pop-Ups in der Regel geblockt werden, ist das also eine klare Dissonanz zwischen den Erwartungen des Nutzers (Ein neues Fenster öffnet sich) und der Reaktion des Systems (Es wird nur ein kleiner Hinweistext angezeigt). Da Pop-Ups auf modernen Seiten im Grundsatz vermieden werden, stellt dies einen groben Fehler dar, da es Nutzer dazu bringen kann, aufgrund von fehlendem Feedback abzubrechen (Kategorie~4).
}
{Hier sollte auf die Verwendung von Pop-Ups grundsätzlich verzichtet werden.
}\label{prob:start:popup}

\problem{Dynamisches Seitenmenü} 
{Das Seitenmenü ist dynamisch an das Header-Menü gekoppelt. Das bedeutet, wenn der Nutzer durch das Header-Menü auf eine andere Seite wechselt, ändern sich die angebotenen Funktionen im Seitenmenü. Das ist äußerst problematisch, da für den Nutzer ohne ersichtlichen Grund wichtige Funktionen, wie \emph{Registrieren} plötzlich nicht mehr zur Verfügung stehen. Es ist auch nicht ersichtlich, wie es möglich ist, diese Funktionen wieder zu aktivieren.
Zusätzlich ist das Seitenmenü auf allen Seiten außer der Startseite völlig funktionslos. 
}
{Das plötzliche Verschwinden wichtiger Funktionen, kann dazu führen, dass der Nutzer den Use-Case nicht beenden kann und abbricht (Kategorie~4).
}
{Das Seitenmenü sollte statisch bleiben und auf jeder der Unterseiten die gleichen Funktionen anbieten.
}\label{prob:start:seitenmenue}

\subsubsection*{Fazit: Seitenmenü}
Das linke Seitenmenü hat sich als fehlerhaft und überflüssig gezeigt. 
Die redundanten Funktionen \emph{Alumni-Portal}, \emph{Portal in Englisch} und \emph{Start} sollten nur noch im Header-Menü angezeigt werden. Die Funktionen \emph{Logindaten vergessen}, \emph{Registrieren} und \emph{Freischalten} wären besser im Anmeldefenster untergebracht und das Impressum sollte an den Fuß der Seite.
Somit wäre jede Funktion des Seitenmenüs ausgelagert und das Menü kann entfernt werden, wie bereits in Abschnitt \ref{prob:start:menues} vorgeschlagen.

%Alex
\usecasepart{Registrierung}

\subsubsection*{Positive Beobachtungen}
... 

\problem{X}
\problem{Y}
\problem{Z} 
%Armin
\usecasepart{Freischaltung}
\label{subsec:freischaltung}

\subsubsection*{Positive Beobachtungen}
\label{subsubsec:freischaltungpositiv}
Mail ist sehr freundlich gehalten\newline
Beim Absenden der Daten erhält der Nutzer durch drehende Zahnräder visuelles Feedback, dass seine Daten gerade verarbeitet werden.

%Probleme in der Mail
\problem{Anrede in der Mail fehlt}

\descript{
Beim Erhalt der Mail fällt sofort auf, dass eine Anrede fehlt. Stattdessen steht als \glqq Ersatz\grqq für die Anrede nur der Nachname da.
}
\category{
Bei einer offiziellen Seite erwartet der Nutzer die Einhaltung gewisser Konventionen, welche hier nicht gegeben ist. Die fehlende Anrede vermittelt zudem einen Mangel an Seriosität, was dem Vertrauen in die Seite abträglich ist. Der Nutzer könnte sich dadurch stark verunsichert fühlen und in Zweifel geraten, dass er sich auf einer seriösen Seite befindet. Es handelt sich hierbei also um ein auffälliges Problem (Kategorie~3).
}
\improvement{
Überarbeitung des Mailtextes unter Beachtung allgemeiner sprachlicher Konventionen.
}

\problem{Mail ist nicht formatiert}

\descript{
Neben der fehlenden Anrede fällt auch sofort die fehlende Formatierung der Mail ins Auge. Am Anfang der Mail drängt sich ein Bild, dessen Beitrag zum Inhalt der Mail sich nicht sofort erschließt, zwischen die \glqq Anrede\grqq ~und den eigentlichen Text. Außerdem beginnt der Text sehr weit links, was je nach verwendetem Programm zu Problemen bei der Lesbarkeit führt.
}
\category{
Die Mängel bei der Formatierung führen nicht dazu, dass der Nutzer seinen Arbeitsablauf abbrechen muss. Es dauert nur etwas länger, sich zurechtzufinden und die Mail zu lesen. Es ist daher nur ein leichtes Problem (Kategorie~2).
}
\improvement{
Hier ist eine Überarbeitung der Formatierung, insbesondere des HTML-Anteils, nötig. Des Weiteren sollte man das Bild entweder anders positionieren, oder eventuell auch ganz weglassen.
}

\problem{Erstmalige Erwähnung von stud\hbox{-}mail-Adressen}

\descript{
In der Mail wird zum ersten Mal im gesamten Registrierungsprozess erwähnt, dass Studenten keine stud\hbox{-}mail-Adressen als Kontaktadressen verwenden sollen.
}
\category{
Dieses Problem könnte bei Nutzern, die noch Studenten sind, für Verwirrung und zusätzlichen, unnötigen Arbeitsaufwand sorgen. Der Nutzer wird dazu gezwungen, seine gerade gewählte E\hbox{-}Mail-Adresse nach dem Login zu ändern, damit er auch in Zukunft sicher erreichbar ist. Der Nutzer kann dies als behindernd empfinden, da er so lange nicht sein Ziel verfolgen kann und durch die zusätzliche Arbeit abgelenkt wird. Es handelt sich also um ein störendes, auffälliges Problem (Kategorie~3).
}
\improvement{
Dieses Problem lässt sich dadurch beheben, dass bereits bei der Eingabe der E\hbox{-}Mail-Adresse zur Registrierung deutlich darauf hingewiesen wird, dass diese Adressen nicht verwendet werden sollen. Das Eingabefeld für die E\hbox{-}Mail-Adresse kann zusätzlich noch mit einer Validierung versehen werden, um sicherzustellen, dass keine stud\hbox{-}mail-Adressen verwendet werden.
}

\problem{Link zur Freischaltung in der Mail}

\descript{
Der Link führt zur gleichen Seite wie der Menüpunkt \glqq Freischalten\grqq ~im Alumni-Portal. Dort gibt es allerdings noch einen Untermenüpunkt \glqq Freischalten\grqq ~mit einer anderen URL im Menü  \glqq Registrieren\grqq.
}
\category{
Das Problem wird einem normalen Nutzer nicht auffallen, da er einfach dem Link in der E\hbox{-}Mail folgt und damit auf der Freischaltungsseite landet (Kategorie~0).
}
\improvement{
Es ist kein Problem, das dem Nutzer auffällt oder ihn in seiner Arbeit beeinträchtigt, dennoch ist es sinnvoll, die Links zur Freischaltung zu vereinheitlichen. Der Link zu \texttt{www.alumni.uni-wuerzburg.de} kann an dieser Stelle weggelassen werden, da es für den Nutzer einen Umweg bedeutet, und er sich erst auf der Seite zurechtfinden muss, um dann den Weg zum Alumni-Portal zu finden.
}

%Probleme auf der Webseite zur Freischaltung
\problem{Button für mailto-Links}

\descript{

}
\category{
text  (Kategorie~2).
}
\improvement{

}

\problem{Eingabe der E-Mail-Adresse}

\descript{

}
\category{
text  (Kategorie~3).
}
\improvement{

}

\problem{Händische Eingabe des Freischaltcodes}

\descript{

}
\category{
text  (Kategorie~2).
}
\improvement{

}

\problem{Wahl von Nutzername und Passwort bei der Freischaltung}

\descript{

}
\category{
text  (Kategorie~2).
}
\improvement{

}

\problem{Verfügbarkeit des gewählten Nutzernamens}

\descript{

}
\category{
text  (Kategorie~2).
}
\improvement{

}

\problem{Grüner Warntext bei der Wahl des Passworts}

\descript{

}
\category{
text  (Kategorie~4).
}
\improvement{

}

\problem{Längenbeschränkung des Passwortfelds}

\descript{

}
\category{
text  (Kategorie~4).
}
\improvement{

}

\problem{Validierung erst beim Absenden der Daten}

\descript{

}
\category{
text  (Kategorie~2).
}
\improvement{

}

\problem{Speichern der Daten dauert sehr lange}

\descript{
Man muss nach dem Klick auf Speichern lange warten (4-5 Sekunden), bis die Daten gespeichert wurden und man sich einloggen kann. Hier ist jedoch positiv zu vermerken, dass der Nutzer in dieser Zeit Feedback erhält (siehe \ref{subsubsec:freischaltungpositiv}).
}
\category{
Da der Nutzer Feedback erhält und nur ein paar Sekunden warten muss, ist es kein Problem (Kategorie~0).
}
\improvement{
Hier ist es sicher möglich, Verbesserungen an der darunter liegenden Infrastruktur (Datenbank etc.) vorzunehmen, um die Verarbeitungszeit zu verkürzen.
}
%Bene
\newpage
\usecase{Wiederherstellung der Logindaten}
Der Nutzer hat nach einer gewissen Zeit seine Logindaten vergessen. Nun möchte er nach dem Abrufen der Seite versuchen seine Logindaten wiederherzustellen.

\usecasepart{Suchen des Wiederherstellungsformulars auf der Startseite}

\problem{\enquote{Logindaten vergessen} an unüblicher Position}{
Der üblicherweise erste Schritt beim Wiederherstellen der Logindaten ist die Suche nach dem Login-Bereich. Erwartungsgemäß befindet sich dort eine Weiterleitung mit der Möglichkeit das Passwort wiederherzustellen. Ein Nutzer der diese Funktion auf der Webseite des Alumni-Portals nutzen möchte findet keinen solchen Hinweis im blau hinterlegten Login-Bereich (siehe Abbildung\ref{fig:Wiederherstellung_1}). Erst beim Zweiten oder Dritten Blick, je nachdem welche Navigationsleiste der Nutzer zuerst durchsucht, stößt er auf den Hinweis \enquote{Logindaten vergessen}. Bereits diese Tatsache wirkt abschreckend auf den Nutzer und beeinträchtigt den Arbeitsfluß.
}{
Erwartungsgemäß ist das Wiederherstellen der Logindaten zeitaufwändig. Wird ein Nutzer bereits beim Finden dieser Funktion auf der Webseite behindert kann es schnell zur Frustration kommen. (Kategorie 3)
}{
Die Option der Logindatenwiederherstellung gehört definitiv zum Bereich der Anmeldung auf einer Webseite. Aus diesem Grund ist diese Option üblicherweise auch im direkten, meist abgegrenzten Bereich, des Anmeldeformulars zu finden. Das Anmelden, Registrieren und die Passwortwiederherstellung fallen unter dieselbe Funktionsdomäne und sollten aus deisem Grund gemeinsam an einer Stelle gruppiert werden.
%\footnote{TODO:HCI 3rd Edition Page:132}
}

%#########################################################################
\usecasepart{Ausfüllen des Wiederherstellungsformulars}

\begin{figure}
	\centering
		\includegraphics[width=\textwidth]{figures/Wiederherstellung_1.jpg}
		\caption{Wiederherstellungsformular für die Logindaten}
	\label{fig:Wiederherstellung_1}
\end{figure}

\subsubsection*{Positive Beobachtungen}
Der Bereich zum Wiederherstellen der Logindaten ist überschaulich gehalten. In zwei kurzen Textabschnitten wird dem Nutzer erklärt wie der Vorgang abläuft und was er dabei zu machen hat. Das Formular hat nur ein einziges Eingabefeld und wirkt aufgeräumt. Der Nutzer hat nicht viel Spielraum für falsche Interpretationen.%Das einzelne Eingabefeld zusammen mit der Beschreibung lässt somit keinen Spielraum für Fehlinterpretationen, auch bei potentiell unerfahrenen Nutzern. 

\problem{Text über den Rand hinaus geschrieben}{
Die Beschreibung zur Logindatenwiederherstellung wird über die Webseitenabgrenzung hinaus weiter geführt (siehe Abbildung\ref{fig:Wiederherstellung_1}). Die fehlenden Formatierung des Textes wirkt unsauber und unprofessionell.
}{
Probleme solcher Art stören nicht direkt die Nutzung der Webseite haben aber einen anderen negativen Effekt. Es stellt beim Nutzer die Seriosität des Portals in Frage, vor allem wenn an anderer Stelle nach der Bankverbindung gefragt wird. Das ganze Auftreten des Alumni-Portals wird durch selche Formatierungsfehler in ein schlechtes Licht gerückt. (Kategorie 1)
}{
Bei der Gestaltung von Texten sollte mehr Wert auf die Formatierung gelegt werden. Bestimmte Bereiche in denen Text stehen kann müssen unbedingt auch eingehalten werden.
}
\problem{Formularfeld nicht beschrieben}{
Das Formuarfeld für die Email-Adresse hat weder eine Beschreibung darüber, noch einen Platzhalter innerhalb des Feldes (siehe Abbildung\ref{fig:Wiederherstellung_1}). Es geht nur aus dem Text hervor welche Eingabe hier vom Nutzer verlangt wird.
}{
Die beiden beschreibenden Texte zur Logindatenwiederherstellung erklären zwar den Nutzen des Inputfeldes, auf den ersten Blick wird dieser allerdings nicht ersichtlich. Warum im Vergleich zum Registrierungsformular an dieser Stelle die Beschreibung entfernt wurde ist fraglich. Für den Nutzer ist somit auf den ersten Blick unklar ob hier eine Email-Adresse, Benutzername oder eine gänzlich andere Information verlangt wird.(Kategorie 1)
}{
Das Formularfeld sollte um eine passende Beschreibung ergänzt werden, entweder mit einem Textfeld darüber oder als Platzhalter innerhalb.
}
\problem{\enquote{Senden} Button mit Diskettensymbol}{
Der Button zum Absenden des Wiederherstellungsformulars enthält ein Diskettensymbol (siehe Abbildung\ref{fig:Wiederherstellung_1}). 
}{
Üblicherweise wird das Diskettensymbol bei Programmen dazu genutzt die gerade bearbeiteten Daten, meist beim Nutzer selbst, zu speichern. In dem Kontext dieser Webseite steht das Symbol für das Speichern der vom Nutzer eingegebenen Daten in der Datenbank des Alumni-Portal Servers. Die Symbolik könnte bei einem Nutzer, dem dieser Hintergrund nicht bekannt ist, für Verwirrung sorgen. (Kategorie 1)
}{
Eine Beschriftung des Buttons der die Funktion des Formulars knapp beschreibt wäre hier angebracht. Beschriftungen die beispielsweise \enquote{Neues Passwort anfordern} oder \enquote{Neue Logindaten anfordern} lauten wären sinnvoller.
}
\problem{Option für \enquote{Email-Adresse erkennen}}{
Die Webseite bietet zwar die Möglichkeit dem Nutzer Logindaten, also Benutzername und Passwort, zukommen zu lassen. Allerdings gibt es keine Option die zur Anmeldung verwendete Email-Adresse erkennen zu lassen. Einige Webseiten bieten Ihren Nutzern diesen Service an.
}{
Ein Ziel des Alumni Portals ist es mit Kommilitonen und Dozenten auch noch Jahre nach dem Abschluss in Kontakt bleiben zu können. Da sich in dieser Zeit auch einmal die Email-Adresse ändern kann ist es sinnvoll einen solchen Sercive zu bieten, auch wenn es sich hierbei per se nicht um Problem handelt. (Kategorie 0)
}{
Unter der Option \enquote{Logindaten vergessen} befindet sich ein zusätzliches Formular welches es dem Nutzer ermöglicht eine Email-Adresse zu einem Benutzernamen erkennen zu lassen. Dabei wird eine Email an die mit dem Benutzernamen verknüpfte Adresse versendet.
}

\begin{figure}
	\centering
		\includegraphics[width=\textwidth]{figures/Wiederherstellung_2.jpg}
		\caption{Feedback nach Absenden des Wiederherstellungsformulars}
	\label{fig:Wiederherstellung_2}
\end{figure}

\problem{Feedback Meldung nicht konsistent mit anderen Meldungen}{
Nach dem Absenden des Formualrs, zur Wiederherstellung der Logindaten, wird die gesamtes Seite neu geladen. An Stelle des Absendeformulars erscheint eine fett gedruckte Benachrichtigung darüber, dass das Kennwort versendet wurde. Zusätzlich wird dem Nutzer mit einem grünen Text für seine Anfrage gedankt. In einer dritten Zeile Text wird in kleinerer Schriftgröße als den anderen beiden Zeilen darauf hingewiesen, dass das neue Kennwort in Kürze im Postfach des Nutzers vorzufinden ist. In Abbildung\ref{fig:Wiederherstellung_2} sind diese drei unterschiedlichen Feedback nachrichten zu sehen.
}{
Feedback ist im Allgemeinen sehr wichtig für einen Nutzer, da er damit über den Status seiner Aktion mit dem System informiert werden kann. Im Fall der Passwortwiederherstellung erhält der Nutzer drei Nachrichten mit jeweils unterschiedlicher Darstellung des Textes. Darüber hinaus unterscheidet sich die Gestaltung anderer Feedback-Nachrichten der selben Webseite von Fall zu Fall. Das kann für Verwirrung beim Nutzer führen, da nicht ersichtlich ist ob es einen Grund für die unterschiedlichen Darstellungen gibt. (Kategorie 1)
}{
Positives sowie negatives Feedback vom System sollte konsistent sein, das heißt es soll die selbe Art und Weise einer Nutzerbenachrichtigung für die gesamte Webseite genutzt werden. Der Nutzer sollte unmissverständlich darüber in Kenntnis gesetzt werden, ob eine Interaktion mit der Webseite funktioniert hat und wenn nicht worin der Fehler lag.
}

%#########################################################################
\usecasepart{Auslesen der Logindaten aus der Email}
Das Formular wurde vom ausgefüllt und abgesendet nun wartet der Nutzer auf die Email mit den neu generierten Logindaten. 

\subsubsection*{Positive Beobachtungen}
Die Email mit neuen Logindaten wird sehr schnell zugestellt. In der Regel war bereits beim Einloggen in den Email-Account die Email vorhanden.

\problem{Formale Anrede in E-Mail fehlt}{
Wurde das \enquote{Logindaten vergessen} Formular ausgefüllt, so erhält der Nutzer eine E-Mail mit den neu generierten Zugangsdaten. Bei dieser Email fällt sofort auf, dass die gesamte Anrede fehlt. Die Email beginnt, ohne jegliches Grußwort, mit \enquote{Ihre Zugangsdaten lauten wie folgt:}.
}{
Eine offizielle Webseite, wie in diesem Fall das Alumni-Portal, sollte mehr Wert auf die Umgangsform mit Ihren Nutzern legen. Diese Email verstärkt das Bild von Unprofessionalität und Unseriosität und ist einzustufen in der Kategorie der auffälligen Probleme. (Kategorie 3)
}{
Die Email sollte, passend zum Nutzer, eine persönliche, formale Anrede enthalten.
}
\problem{Fehlender Link zur direkten Anmeldung}{
In der Email werden dem Nutzer die neu generierten Logindaten mitgeteilt, ein Link mit dem der Nutzer auf die Webseite geleitet wird fehlt allerdings.
}{
Gewöhnlich befindet sich in einer Email mit der Userdaten oder Freischaltcodes versendet werden ein Link, mit dessen Hilfe der Nutzer direkt auf die Webseite gelangt. Das ermöglicht einerseits den unnötigen Aufwand des Benutzers die Webseite manuell aufzurufen, andererseits könnten generierte Codes an den Link gehängt werden. Damit würde gegebenfalls das manuelle Übertragen eines Freischaltcodes oder Passworts die Bedienung für den Nutzer erleichtern. (Kategorie 2)
}{
Eine simple Verbesserung wäre ein Link mit dem ein Nutzer direkt zum Anmeldeformular kommt. Auch zu denken wäre eine Umsetzung in der ein Nutzer durch den Link direkt auf ein Formular geleitet wird in dem er ein neues Passwort setzen kann.
}
%#########################################################################
\usecasepart{Anmelden mit neuen Logindaten}
Hat der Nutzer die Email mit den neu generierten Logindaten erhalten, so ruft er die Webseite des Alumni-Portal auf um sich damit das erste Mal einzuloggen. 

\subsubsection*{Positive Beobachtungen}
Das Anmelden mit den neuen Logindaten läuft ohne Probleme, es gibts nichts zu beanstanden.

%#########################################################################
\usecasepart{Setzen eines neuen Passworts}
Nach dem ersten Anmelden mit den neuen Logindaten wird ein Formular angezeigt mit dem ein neues Kennwort gesetzt werden kann. 

\begin{figure}
	\centering
		\includegraphics[width=\textwidth]{figures/Wiederherstellung_3.jpg}
		\caption{Formular zur Passwortänderung}
	\label{fig:Wiederherstellung_3}
\end{figure}

\subsubsection*{Positive Beobachtungen}
Die direkte Weiterleitung auf das Formular zum Ändern des Passwortes erleichtert dem Nutzer die Bedienung.

\problem{Fehlende Hinweise und live Validierung zu gültigen Passwörtern}{
Dem Formular zum Ändern des Passworts fehlt ein Hinweis darüber, welche Form ein Passwort haben muss oder darf. Es gibt weder ein Hinweis in Textform, noch gibt es irgendeine Art von live Validierung des Eingabefeldes (siehe Abbildung\ref{fig:Wiederherstellung_3}).}{
Ein Grund für das Wiederherstellen der Logindaten könnte darin bestehen, dass die Hinweise zu gültigen Kennwörtern bereits bei der Registrierung nicht ausreichend war\footnote{Beispielsweise der fehlende Hinweis auf die Längenbegrenzung des Passworts auf 17 zeichen.} Genau der selbe Fehler wird an dieser Stelle wiederholt. Bei dem Formular zur Änderung des Passworts steht kein Hinweis wie eine gültige Passworteingabe auszusehen hat. Dieser Fakt ist äußerst ärgerlich für jeden Nutzer und mündet in Frustration. Ebenfalls wurde es versäumt eine Eingabevalidierung der Felder durchzuführen während das Passwort eingegeben wird bzw. das Inputfeld verlassen wird. Lediglich beim Absenden des Formulars werden die Felder auf fehlerhafte Eingaben geprüft und der Nutzer darauf hingewiesen. Gerade bei Passwörtern ist das viel zu spät und damit nicht ausreichend. Störend und zugleich frustrierend ist dabei auch, dass die Eingabefelder nach dem erfolglosen Absenden geleert werden. (Kategorie 3)
}{
Abhilfe würde ein einheitlicher Hinweistext schaffen, wann ein Passwort vom System akzeptiert wird und wann nicht. Ein solcher Hinweis könnte bei Eingaben des Nutzers leicht hervorgehoben werden sobald dieser in das jeweilige Formularfeld klickt. Damit würde der Nutzer bei der Passwortwahl direkt darauf aufmerksam gemacht werden. Die fehlende Eingabevalidierung sollte ebenfalls implementiert werden. Durch kleine Hinweise kann der Nutzer direkt erkennen was er bei der Passwortwahl beachten muss oder vielleicht vergessen hat.}

\problem{Feedback bei Passwortänderung}{
Das Feedback bei der Passwortänderung erscheint zwischen der oberen Navigationleiste und dem Formular zur Passwortänderung. Der Einzeiler zur erfolgreichen Kennwortänderung entspricht wiederum nicht den anderen Feedbackmeldungen.}{
Wie bereits bei den anderen Feedbackmeldungen bemängelt zeigt sich auch hier wieder eine Inkonsistenz zum Rest der Webseite. (Kategorie 1)
}{
Ein mögliches Feedback sollte auch an dieser Stelle den Nutzer über die erfolgreiche Passwortänderung hinweisen. Allerdings konsistent mit allen anderen Meldungen des Systems an den User. 
}

















%Thomas
\usecasepart{Suchen eines Nutzers}

\subsubsection*{Positive Beobachtungen}
... 

\problem{X}
\problem{Y}
\problem{Z} 

\section{Fazit}\label{sec:fazit}
Für die Evaluation wurde ein typischer Use-Case gewählt. Er beinhaltet das Besuchen der Startseite, gefolgt von einer Registrierung inklusive Freischaltung und/oder Zurücksetzen des Passworts sowie die Suche nach ehemaligen Kommilitonen. 
Im Rahmen dieses Use-Case haben sich bereits schwerwiegende Probleme hinsichtlich User-Interface und Kommunikation in Form von Feedback mit dem Nutzer gezeigt. Das Portal bleibt dabei in mehreren Aspekten hinter dem aktuellen Standard moderner Webseiten zurück. 

Zusammenfassend lässt sich sagen, dass die Webseite des Alumni-Portals deutliche Schwächen aufweist. Dissonanzen zwischen den Erwartungen des Nutzers und der Reaktion der Webseite sind ein häufiges Problem.
Zurückführen lässt sich dies überwiegend auf die mangelnde Kommunikation von Informationen und Systemfeedback an den Nutzer.
Zum Teil werden gar Funktionen beworben, die schlichtweg nicht Bestandteil der aktuellen Implementierung der Webseite sind.

Abschließend hinterlässt das Alumni-Portal der Universität Würzburg aufgrund der aufzeigten Mängel einen unprofessionellen und unseriösen Eindruck. 
Da das Portal die Universität und den Alumni-Verein repräsentiert, hat ein qualitativ schlechter Internetauftritt auch einen negativen Einfluss auf die Außenwirkung und den Ruf der Universität Würzburg. 


%bisher nur die bib aus dem example report
%\bibliography{report}

\newpage
\appendix

\section{Designalternativen}
BLABLA, weitere Sachen die uns aufgefallen sind, die aber nicht Bestandteil unseres Walkthroughs waren



\end{document}



