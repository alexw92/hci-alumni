\documentclass[fontsize=12pt,a4paper]{scrartcl}

\usepackage[utf8]{inputenc}
\usepackage[USenglish,american]{babel}
\usepackage{hyperref}
\usepackage{graphicx}
\usepackage{caption}
\usepackage{subcaption}
\usepackage{enumitem}
\usepackage{tabularx}

\bibliographystyle{apalike}

\usepackage{wrapfig}
\usepackage{breakcites}
\usepackage[autostyle=true,german=quotes]{csquotes}

\begin{document}
\titlehead{
\includegraphics[scale=1.015]{figures/uni-wb-hci-header}
}

\subject{Wintersemester 2014/2015\\ Einführung in die Mensch-Computer-Interaktion}

\title{BEISPIEL Report}

\subtitle{Cognitive Walkthrough}

\publishers{
Benedikt Pfaff, Matrikelnummer 2060170\\
Armin Beutel, Matrikelnummer 1790705\\
Johannes Grohmann, Matrikelnummer 1234567\\
Thomas Handwerker, Matrikelnummer 1995289\\
Alexander Werthmann, Matrikelnummer 1234567\\[3em] 
\normalsize Supervisors: Chris Zimmerer, Kristof Korwisi}

\maketitle

\setcounter{page}{0}
\thispagestyle{empty}

\newpage

\pagenumbering{Roman}
\setcounter{page}{1}


\tableofcontents

\newpage

\setcounter{page}{1}
\pagenumbering{arabic}

\section{Einleitung}
Dieses Dokument soll eine Evaluation enthalten, welche sich mit dem Use Case: Registrierung auf der Alumni Webseite der Universität Würzburg beschäftigt. Zum Zweck der Evaluation wird die Methode des Cognitive Walkthrough angewendet. Die Methode unterteilt sich in drei Phasen: Vorbereitung, Analyse sowie Zusammenfassung von Ergebnissen und Verbesserungsvorschlägen.

%############################################################################################################################

\section{Vorbereitung}
\subsection{Potentielle User}
Der Kreis der potentiellen User umfasst alle Studierenden und Absolventen der Uni Würzburg, sowie Professoren, Dozenten und Mitarbeiter der Universität. Außerdem gehören dazu, ausländische (nicht deutschsprachige) und inländische Gaststudenten sowie Interessierte. Grundsätzlich wird davon ausgegangen, dass ein potentieller User Akademiker ist und/oder einen höheren Bildungsabschluss hat. Desweiteren wird unterstellt, dass der User bereits Erfahrung im Umgang mit dem Browser, dem Abrufen und Schreiben von e-Mails und dem Registrieren auf Webseiten hat. 


\subsection{Aufgabenbeschreibung}
Für die Evaluation wird das Registrieren auf der Webseite der Alumni Uni Würzburg gewählt. Dabei sollen alle Schritte bis zur erfolgreichen Registrierung durchgeführt werden. Zudem soll ein Anmeldeversuch nach erfolgreicher Registrierung durchgeführt werden.
Die Registrierung ist in folgende Schritte untergliedert:
\begin{enumerate}
	\item Aufrufen der Alumni Webseite
	\item Zurechfinden auf der Webseite und Finden des Registrierungsformulars
	\item Ausfülledes ersten Teiles des Registrierungsformulars
	\item Absenden des ersten Formulars
	\item Ausfüllen des zweiten Teiles des Registrierungsfomulares
	\item Absenden des zweiten Formulares
	\item weitere Schritte...
\end{enumerate}
 
 
\subsection{Mögliche Aktionen und Effekte}
\begin{enumerate}
	\item 
	\begin{itemize}
		\item Auswahl !Registrieren! in der linken Navigations-Leiste 
	\end{itemize}
	\item 
	\begin{itemize}
		\item Eingabe Persönlicher Daten
		\item Eingabe der Adresse
		\item Eingabe der Heimatadresse
		\item Eingabe Berufliche Einbindung
		\item Auswahl Alumni Angebot
		\item Auswahl zur kostenpflichtigen Mitgliedschaft
		\item Auswahl !Wie haben Sie von der Alumni Würzburg erfahren?!
		\item Markierung zur Einverständniserklärung der Datenschutzerklärung
		\item Markierung zur Einverständniserklärung der Nutzung Personenbezogener Daten
	\end{itemize}
	\item 
	\begin{itemize}
		\item Bestätigung durch den Button !Weiter!
		\item Abbruch durch den Button !Abbruch! 
	\end{itemize}
	\item 
	\begin{itemize}
		\item Eigabe der Daten des Zweiten Formulars...
	\end{itemize}
	\item 
	\begin{itemize}
		\item Absenden des zweiten Formulars...
	\end{itemize}
	\item 
	\begin{itemize}
		\item weitere Schritte...
	\end{itemize}
\end{enumerate}

%############################################################################################################################

\section{Analyse}
\subsection{Evaluationsergebnisse}
Die im Rahmen des Cognitive Walkthrough gemachten positiven Beobachtungen und detektierten Probleme bezüglich Design und Bedienbarkeit werden in den nachfolgenden Abschnitten aufgelistet.
Dabei werden pro Problem jeweils die genaue Problembeschreibung, eine Kategorisierung des Problems sowie ein Verbesserungsvorschlag genannt.
Tabelle \ref{tbl:schwere} listet die in diesem Zusammenhang verwendeten Kategorien auf. 
Die Unterteilung der gefundenen Probleme erfolgt anhand der Aufgabenbereiche in denen sie aufgetreten sind. 

\begin{table}
	\centering\begin{tabular}{|c|l|}
		\hline
		\textbf{Kategorie} & \textbf{Bedeutung} \\
		\hline
		0 & Kein Problem \\
		1 & Ein Problem \\
		2 & Zwei Problem \\
		3 & Drei Problem \\
		4 & Vier Problem \\
		\hline
	\end{tabular}
	\caption{Problemschweregrade\label{tbl:schwere}}
\end{table}


\subsection{Walkthrough}
\subsubsection{Problem:...}

\subsection{Zusammenfassung von Ergebnissen}
Für die Evaluation wird ein Beispiel aus dem Bereich des Kündigungsschutzes gewählt. Es handelt sich um die telefonische Beratung einer Klientin, bei welcher die beratende Person die in Anhang \ref{sec:infos} aufgeführten Informationen erhält. Vom Berater/von der Beraterin sind zwei Fragen zu beantworten:

%############################################################################################################################

\section{Designalternativen}
BLABLA



\end{document}



