\section{Beispiel-Walkthrough}
\subsection{Aufgabenbeschreibung}
F�r die Evaluation wird das Registrieren auf der Webseite der Alumni Uni W�rzburg gew�hlt. Dabei sollen alle Schritte bis zur erfolgreichen Registrierung durchgef�hrt werden. Zudem soll ein Anmeldeversuch nach erfolgreicher Registrierung durchgef�hrt werden.
Die Registrierung ist in folgende Schritte untergliedert:
\begin{enumerate}
	\item Aufrufen der Alumni Webseite
	\item Zurechfinden auf der Webseite und Finden des Registrierungsformulars
	\item Ausf�lledes ersten Teiles des Registrierungsformulars
	\item Absenden des ersten Formulars
	\item Ausf�llen des zweiten Teiles des Registrierungsfomulares
	\item Absenden des zweiten Formulares
	\item weitere Schritte...
\end{enumerate}

\subsection{M�gliche Aktionen und Effekte}
\begin{enumerate}
	\item 
	\begin{itemize}
		\item Auswahl !Registrieren! in der linken Navigations-Leiste 
	\end{itemize}
	\item 
	\begin{itemize}
		\item Eingabe Pers�nlicher Daten
		\item Eingabe der Adresse
		\item Eingabe der Heimatadresse
		\item Eingabe Berufliche Einbindung
		\item Auswahl Alumni Angebot
		\item Auswahl zur kostenpflichtigen Mitgliedschaft
		\item Auswahl !Wie haben Sie von der Alumni W�rzburg erfahren?!
		\item Markierung zur Einverst�ndniserkl�rung der Datenschutzerkl�rung
		\item Markierung zur Einverst�ndniserkl�rung der Nutzung Personenbezogener Daten
	\end{itemize}
	\item 
	\begin{itemize}
		\item Best�tigung durch den Button !Weiter!
		\item Abbruch durch den Button !Abbruch! 
	\end{itemize}
	\item 
	\begin{itemize}
		\item Eigabe der Daten des Zweiten Formulars...
	\end{itemize}
	\item 
	\begin{itemize}
		\item Absenden des zweiten Formulars...
	\end{itemize}
	\item 
	\begin{itemize}
		\item weitere Schritte...
	\end{itemize}
\end{enumerate}

\subsection{Analyse}
\subsubsection{Evaluationsergebnisse}
Die im Rahmen des Cognitive Walkthrough gemachten positiven Beobachtungen und detektierten Probleme bez�glich Design und Bedienbarkeit werden in den nachfolgenden Abschnitten aufgelistet.
Dabei werden pro Problem jeweils die genaue Problembeschreibung, eine Kategorisierung des Problems sowie ein Verbesserungsvorschlag genannt.
Tabelle \ref{tbl:schwere} listet die in diesem Zusammenhang verwendeten Kategorien auf. 
Die Unterteilung der gefundenen Probleme erfolgt anhand der Aufgabenbereiche in denen sie aufgetreten sind. 

\begin{table}
	\centering\begin{tabular}{|c|l|}
		\hline
		\textbf{Kategorie} & \textbf{Bedeutung} \\
		\hline
		0 & Kein Problem \\
		1 & Ein Problem \\
		2 & Zwei Problem \\
		3 & Drei Problem \\
		4 & Vier Problem \\
		\hline
	\end{tabular}
	\caption{Problemschweregrade\label{tbl:schwere}}
\end{table}


\subsubsection{Walkthrough}
viel Text
\paragraph{Problem 1}
Text
\paragraph{Problem 2}
Text Text