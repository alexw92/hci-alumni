\section{Vorbereitung}
\subsection{Potenzielle Nutzer}
Der Kreis der potenziellen Nutzer umfasst alle Studierenden und Absolventen der Uni Würzburg, sowie Professoren, Dozenten und Mitarbeiter der Universität. Weiterhin gehören hierzu inländische und ausländische/nicht deutschsprachige Gaststudenten sowie Interessierte. Grundsätzlich wird davon ausgegangen, dass ein potentieller Nutzer Akademiker ist und/oder einen höheren Bildungsabschluss hat. Des Weiteren wird unterstellt, dass der Nutzer bereits Erfahrung im Umgang mit einem Browser, dem Abrufen und Schreiben von E-Mails und dem Registrieren auf Webseiten hat. 

%hier allgemeine Aufgabenbeschreibung, was machen wir insgesamt
\subsection{Aufgabenbeschreibung}
Für die Evaluation des Alumni-Portals der Universität Würzburg haben wir uns einen speziellen Use-Case, also eine Aufgabe, die der Nutzer ausführen will, definiert. 
Für die vorliegende Ausarbeitung ist dieser in fünf Teil-Use-Cases aufgeteilt, die im Folgenden ausführlich beschrieben und evaluiert werden. Obwohl die Use-Cases einzeln betrachtet werden, bauen sie sukzessiv aufeinander auf und stellen so eine einzige Aufgabe dar.

\subsubsection*{Nutzerspezifikation}
In unserem Walkthrough gehen wir von einem deutschsprachigen Alumni der Universität Würzburg aus. Zusätzlich wird angenommen, dass keine Farbenblindheit oder sonstige visuelle Beeinträchtigungen vorliegen. 
Der Anwender ruft die Webseite mit einem Desktop-PC (Auflösung 1920$\times$1080) auf. 
Das Geschlecht des Nutzers ist nicht weiter festgelegt. Im Folgenden soll der Terminus \emph{Nutzer} sowohl für einen weiblichen als auch einen männlichen Nutzer stehen.

\paragraph{Use-Case 1: Startseite}\quad\\
Der Nutzer hat vom Alumni-Portal der Universität gehört und surft die Webseite \url{https://uni-wuerzburg.alumnionline.de} an. Da er sich zum ersten Mal auf dieser Seite befindet, nimmt er sich einen Augenblick Zeit, um sich einen Überblick über den Aufbau und die Funktionen des Portals zu verschaffen.
Hier ergeben sich folgende Aufgabenteile:
\begin{enumerate}

		\item Betrachten der Startseite
		\item Betrachten der Menüs
\end{enumerate}

\paragraph{Use-Case 2: Registrieren}\quad\\
Nachdem sich der Nutzer einen Überblick über die Seite verschafft hat und nun über verschiedenen Funktionen, die das Alumni-Portal bietet teils gut, teils weniger gut informiert wurde, entschließt er sich nun für eine kostenlose Registrierung, um das Alumni-Portal in größerem Umfang zu nutzen.

Das führt zu den folgenden Aufgabenteilen:
\begin{enumerate}
		\item Ausfüllen des Eingabeformulars
		\item Abschicken des Eingabeformulars
\end{enumerate}

\paragraph{Use-Case 3: Freischalten}\quad\\
Der Nutzer erhält nach einiger Wartezeit eine E\hbox{-}Mail mit seinem persönlichen Freischaltcode. Diesen benötigt er für die Verifizierung und Aktivierung seines Accounts. Die Mail enthält einen Link, der vom Nutzer angeklickt wird und ihn direkt zur Freischaltung auf dem Portal führt. Hier gibt der Nutzer die geforderten Daten ein und schließt damit die Aktivierung seines Accounts ab.

Damit ergeben sich folgende Aufgabenteile:
\begin{enumerate}
		\item Empfangen und Lesen der Mail
		\item Ausfüllen der Webseite zur Freischaltung
		\item Absenden der Daten
\end{enumerate}


\paragraph{Use-Case 4: Wiederherstellung der Logindaten}\quad\\
Eine gewisse Zeit ist vergangen und der Nutzer hat seine Logindaten vergessen. Um das Portal trotzdem weiter nutzen zu können, entscheidet er sich dazu eine Passwortwiederherstellung durchzuführen. 
Dabei geht er folgende Schritte durch:
\begin{enumerate}
\item Suchen des Wiederherstellungsformulars
\item Ausfüllen des Wiederherstellungsformulars
\item Auslesen der Logindaten aus der E-Mail
\item Anmelden mit den neuen Logindaten
\item Setzen eines neuen Passwortes
\end{enumerate}


\paragraph{Use-Case 5: Besucher sucht einen ehemaligen Kommilitonen}\quad \\
Der Nutzer möchte nun mit Hilfe der bereitgestellten Suchfunktion einen ehemaligen Kommilitonen aus der früheren Studienzeit finden. Dabei sind ihm lediglich die allgemeinen Informationen, wie Vor- und Nachname des Studienkollegen, als auch Studiengang und Abschlussjahr, bekannt. 
Die Suchmaske, sowie erweiterte Suchoptionen stehen dem Anwender bei seinem Vorhaben zur Verfügung. Die darauffolgende Trefferliste bietet dem Nutzer alle zur Suchanfrage relevanten Alumni an, die auf dem Portal registriert sind.

Auch dieser Use-Case lässt sich in mehrere Aufgabenteile aufteilen:
\begin{enumerate}
	\item Anmeldung am Alumni-Portal
	\item Aufrufen der Suchmaske über das Menü
	\item Auswahl der gewünschten Suchmaske
	\item Alumni über einen Suchbegriff finden
	\item Alumni über erweiterte Suchoptionen finden
	\item Alumni über die Studiengangliste finden
	\item Sichten der Trefferliste einer Suche
\end{enumerate}



In den folgenden Kapiteln werden die einzelnen Use-Cases in mögliche Aktionen und Effekte aufgeteilt, um exaktere Analysen zu gewährleisten.

\subsection{Mögliche Aktionen und Effekte}
Um die zuvor beschriebenen Aufgaben durchführen zu können stehen dem Nutzer verschiedene Aktionen zur Verfügung, die er ausführen kann. Welche diese sind und welche Effekte dabei erzielt werden ist im Folgenden aufgelistet.

\begin{itemize}
\item Klick auf einen Menüpunkt zum Aufrufen einer Unterseite
\item Ausfüllen unterschiedlicher Formularfelder
\item Aufruf einer externen Seite durch Klick auf einen Hyperlink
\item Absenden der Formulardaten durch Klick auf einen Button
\item Bereitstellen unterschiedlicher Auswahlmöglichkeiten durch Öffnen eines Dropdown-Menüs
\item Markieren von Kontrollkästchen (Checkboxen)
\end{itemize}