\section{Vorbereitung}
\subsection{Potentielle Nutzer}
Der Kreis der potentiellen Nutzer umfasst alle Studierenden und Absolventen der Uni Würzburg, sowie Professoren, Dozenten und Mitarbeiter der Universität. Außerdem gehören dazu inländische und ausländische/nicht deutschsprachige Gaststudenten sowie Interessierte. Grundsätzlich wird davon ausgegangen, dass ein potentieller Nutzer Akademiker ist und/oder einen höheren Bildungsabschluss hat. Des Weiteren wird unterstellt, dass der Nutzer bereits Erfahrung im Umgang mit einem Browser, dem Abrufen und Schreiben von E-Mails und dem Registrieren auf Webseiten hat. 

%hier allgemeine Aufgabenbeschreibung, was machen wir insgesamt
\subsection{Aufgabenbeschreibung}
Für die Evaluation des Alumni-Portals der Universität Würzburg haben wir uns einen speziellen Use-Case, also eine Aufgabe, die der Nutzer ausführen will definiert. 
Für die Ausarbeitung ist dieser in fünf Teil-Use-Cases aufgeteilt, die im Folgenden ausführlich beschrieben und evaluiert werden. Obwohl die Use-Cases einzeln betrachtet werden, bauen sie sukzessiv aufeinander auf und stellen so eine einzige Aufgabe dar.

\subsubsection*{Nutzerspezifikation}
In unserem Walkthrough gehen wir von einem deutschsprachigen Alumni der Universität Würzburg aus. Zusätzlich wird angenommen, dass keine Farbenblindheit oder sonstige visuelle beeinträchtigen vorliegen. 
Außerdem wird die Webseite mit dem Desktop-PC (Auflösung 1920$\times$1080) des Nutzers aufgerufen. 
Das Geschlecht des Nutzers oder der Nutzerin ist nicht weiter festgelegt. Im Folgenden soll der Terminus \emph{Nutzer}, sowohl für einen weiblichen als auch einen männlichen Nutzer stehen.

\paragraph{Use-Case 1: Startseite}\quad\\
Der Nutzer hat vom Alumni-Portal der Universität gehört und surft die Webseite \url{https://uni-wuerzburg.alumnionline.de} an. Da er sich zum erstem Mal auf dieser Seite befindet, nimmt er sich einen Augenblick Zeit, um sich einen Überblick über den Aufbau und die Funktionen des Portals zu verschaffen.
Hier ergeben sich folgende Aufgabenteile:
\begin{enumerate}

		\item Betrachten der Startseite
		\item Betrachten der Menüs
\end{enumerate}

\paragraph{Use-Case 2: Registrieren}\quad\\
Der Use-Case ist volle geil

Damit ergeben sich folgende Aufgabenteile:
\begin{enumerate}

		\item Betrachten der Startseite
		\item Betrachten der Menüs
\end{enumerate}

\paragraph{Use-Case 3: Freischalten}\quad\\
Der Nutzer erhält nach einiger Wartezeit eine E\hbox{-}Mail, die den Freischaltcode enthält, den der Nutzer benötigt, um seinen Account freizuschalten. Die Mail enthält einen Link, der vom Nutzer angeklickt wird und ihn direkt zur Freischalteseite führt. Hier gibt der Nutzer die geforderten Daten ein und schließt damit die Freischaltung seines Accounts ab.

Damit ergeben sich folgende Aufgabenteile:
\begin{enumerate}
		\item Empfangen und Lesen der Mail
		\item Ausfüllen der Webseite zur Freischaltung
		\item Absenden der Daten
\end{enumerate}


\paragraph{Use-Case 4: Registrieren}\quad\\
Der Use-Case ist volle geil

Damit ergeben sich folgende Aufgabenteile:
\begin{enumerate}

		\item Betrachten der Startseite
		\item Betrachten der Menüs
\end{enumerate}


\paragraph{Use-Case 5: Besucher sucht einen ehemaligen Kommilitonen}\quad \\
Ein auf dem Portal angemeldeter Nutzer möchte mit Hilfe der bereitgestellten Suchfunktion einen ehemaligen Kommilitonen aus der früheren Studienzeit finden. Dabei sind dem Anwender lediglich die allgemeinen Informationen, wie Vor- und Nachname des Studienkollegen, als auch Studiengang und Abschlussjahr, bekannt. Die Suchmaske, bzw. die erweiterte Suchfunktion der Website stehen dem Anwender zum Starten der Suchanfrage zur Verfügung. Nach erfolgreicher Suche gelangt der User auf das Profil des Kommilitonen. Dort stehen detaillierte Informationen zur gesuchten Person und dessen beruflichen bzw. schulischen Werdegangs. Ebenfalls besteht auf der Profilseite die Möglichkeit einer Kontaktaufnahme via private Nachricht über das Portal.

Damit ergeben sich folgende Aufgabenteile:
\begin{enumerate}
	\item Betrachten der Startseite
	\item Betrachten der Menüs
\end{enumerate}


In den folgenden Kapiteln werden die einzelnen Use-Cases in mögliche Aktionen und Effekte aufgeteilt, um exaktere Analysen zu gewährleisten.

\subsection{Mögliche Aktionen und Effekte}
Um die zuvor beschriebenen Aufgaben durchführen zu können stehen dem Nutzer verschiedene Aktionen zur Verfügung, die er ausführen kann. Welche diese sind und welche Effekte dabei erzielt werden ist im Folgenden aufgelistet.

\begin{itemize}
\item Klick auf Menüpunkt zum Aufrufen einer Unterseite
\item Ausfüllen eines Formularfeldes
\item Aufruf einer externen Seite durch Klicken auf Hyperlink
\item Absenden der Formulardaten durch Klick auf Button
\item Bereitstellen unterschiedlicher Auswahlmöglichkeiten durch Klick auf Dropdown-Menü
\item Markieren einer Checkbox
\end{itemize}