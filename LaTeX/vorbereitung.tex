\section{Vorbereitung}
\subsection{Potentielle Nutzer}
Der Kreis der potentiellen Nutzer umfasst alle Studierenden und Absolventen der Uni Würzburg, sowie Professoren, Dozenten und Mitarbeiter der Universität. Außerdem gehören dazu, ausländische (nicht deutschsprachige) und inländische Gaststudenten sowie Interessierte. Grundsätzlich wird davon ausgegangen, dass ein potentieller User Akademiker ist und/oder einen höheren Bildungsabschluss hat. Des weiteren wird unterstellt, dass der User bereits Erfahrung im Umgang mit dem Browser, dem Abrufen und Schreiben von E-Mails und dem Registrieren auf Webseiten hat. 

%hier allgemeine Aufgabenbeschreibung, was machen wir insgesamt
\subsection{Aufgabenbeschreibung}
Die Evaluation des Alumni-Portal der Universität Würzburg umfasst insgesamt mehrere Aufgaben, die vom Besucher ausgeführt werden können. Für die Ausarbeitung sind im folgenden fünf Use-Cases festgelegt, die im weiteren Verlauf des Walkthrough ausführlich beschrieben und evaluiert werden. 

\subsubsection*{Nutzerspezifikation}
In unserem Walkthrough gehen wir von einem deutschsprachigen Alumni der Universität Würzburg aus. Zusätzlich wird angenommen, dass keine Farbenblindheit oder sonstige visuelle beeinträchtigen vorliegen. 
Außerdem wird die Webseite mit dem Desktop-PC (Auflösung 1920$\times$1080) des Nutzers aufgerufen. 

\paragraph{Use-Case 1: Startseite}\quad\\
Der Use-Case ist volle geil

Damit ergeben sich folgende Aufgabenteile:
\begin{enumerate}

		\item Betrachten der Startseite
		\item Betrachten der Menüs
\end{enumerate}

\paragraph{Use-Case 2: Registrieren}\quad\\
Der Use-Case ist volle geil

Damit ergeben sich folgende Aufgabenteile:
\begin{enumerate}

		\item Betrachten der Startseite
		\item Betrachten der Menüs
\end{enumerate}

\paragraph{Use-Case 3: Registrieren}\quad\\
Der Use-Case ist volle geil

Damit ergeben sich folgende Aufgabenteile:
\begin{enumerate}

		\item Betrachten der Startseite
		\item Ausfüllen des Wiederherstellungsformulars
		\item Auslesen der Logindaten aus der Email
		\item Anmelden mit neuen Logindaten
		\item Setzen eines neuen Passworts
\end{enumerate}


\paragraph{Use-Case 4: Wiederherstellung der Logindaten}\quad\\
Der Use-Case ist volle geil

Damit ergeben sich folgende Aufgabenteile:
\begin{enumerate}

		\item Suchen des Wiederherstellungsformulars auf der Startseite
		\item Betrachten der Menüs
\end{enumerate}


\paragraph{Use-Case 5: Besucher sucht einen ehemaligen Kommilitonen}\quad \\
Ein auf dem Portal angemeldeter Nutzer möchte mit Hilfe der bereitgestellten Suchfunktion einen ehemaligen Kommilitonen aus der früheren Studienzeit finden. Dabei sind dem Anwender lediglich die allgemeinen Informationen, wie Vor- und Nachname des Studienkollegen, als auch Studiengang und Abschlussjahr, bekannt. Die Suchmaske, bzw. die erweiterte Suchfunktion der Website stehen dem Anwender zum Starten der Suchanfrage zur Verfügung. Nach erfolgreicher Suche gelangt der User auf das Profil des Kommilitonen. Dort stehen detaillierte Informationen zur gesuchten Person und dessen beruflichen bzw. schulischen Werdegangs. Ebenfalls besteht auf der Profilseite die Möglichkeit einer Kontaktaufnahme via private Nachricht über das Portal.

Damit ergeben sich folgende Aufgabenteile:
\begin{enumerate}
	\item Betrachten der Startseite
	\item Betrachten der Menüs
\end{enumerate}


In den folgenden Kapiteln werden die einzelnen Use-Cases in mögliche Aktionen und Effekte aufgeteilt, um exaktere Analysen zu gewährleisten.

\subsection{Mögliche Aktionen und Effekte}
Um die zuvor beschriebenen Aufgaben durchführen zu können stehen dem Nutzer verschiedene Aktionen zur Verfügung, die er ausführen kann. Welche diese sind und welche Effekte dabei erzielt werden ist folgend nach UseCase sortiert aufgelistet.

UseCase 1: <use case ..>
bla bla

….

\paragraph{Use-Case 5: Besucher sucht einen ehemaligen Kommilitonen}\quad \\
\begin{itemize}

\item Mit persönlichen Zugangsdaten am Portal anmelden; der Login schlägt aufgrund falscher Eingaben fehl oder aber erfolgt aufgrund korrekter Zugangsdaten
\item Aufrufen der Alumni-Suchfunktion über das horizontale Menü
\item Auswahl der gewünschten Suchmaske im linken Menübereich (Kontaktdaten oder Studiengangs- / Jahrgangsliste)
\item Nach Kontaktdaten:
\item Eingabe eines Suchbegriffs (Vor- und Nachname, Wohnort, etc.) in die Suchleiste eingeben und mit Suchen den Vorgang starten
\item Aktivieren der erweiterten Suchmaske über den Link erweiterte Suche
\item Eingabe bekannter Informationen bzw. Auswahl aus den vorgegebenen Dropdown-Menüs; Suchvorgang wiederum über den Knopf Suchen starten
Abspeichern der ausgefüllten Suchmaske für spätere Suchvorgänge über den Link Suche speichern
Über Suche laden eine im Vorfeld selbst angelegte Suchvorlage laden, um dieser wieder zu verwenden
Nach Studiengangs- / Jahrgangsliste
Eingrenzen der vollständigen Liste durch verschiedene Kriterien möglich: nach Anfangsbuchstaben oder anhand von Beginn- und Abschlussjahr filtern
Auswahl durch Klick auf den gewünschten Studiengang
Durchsuchen der Trefferliste (inklusive vorwärts und rückwärts Blättern mit Hilfe der Pagination-Elemente) und Klick auf den Name des gewünschten Alumni um detaillierte Profilseite aufzurufen
\end{itemize}