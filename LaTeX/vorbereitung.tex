\section{Vorbereitung}
\subsection{Potentielle User}
Der Kreis der potentiellen User umfasst alle Studierenden und Absolventen der Uni Würzburg, sowie Professoren, Dozenten und Mitarbeiter der Universität. Außerdem gehören dazu, ausländische (nicht deutschsprachige) und inländische Gaststudenten sowie Interessierte. Grundsätzlich wird davon ausgegangen, dass ein potentieller User Akademiker ist und/oder einen höheren Bildungsabschluss hat. Desweiteren wird unterstellt, dass der User bereits Erfahrung im Umgang mit dem Browser, dem Abrufen und Schreiben von e-Mails und dem Registrieren auf Webseiten hat. 

%hier allgemeine Aufgabenbeschreibung, was machen wir insgesamt
\subsection{Aufgabenbeschreibung}
Für die Evaluation wird das Registrieren auf der Webseite der Alumni Uni Würzburg gewählt. Dabei sollen alle Schritte bis zur erfolgreichen Registrierung durchgeführt werden. Zudem soll ein Anmeldeversuch nach erfolgreicher Registrierung durchgeführt werden.
Die Registrierung ist in folgende Schritte untergliedert:
\begin{enumerate}
	\item Aufrufen der Alumni Webseite
	\item Zurechfinden auf der Webseite und Finden des Registrierungsformulars
	\item Ausfülledes ersten Teiles des Registrierungsformulars
	\item Absenden des ersten Formulars
	\item Ausfüllen des zweiten Teiles des Registrierungsfomulares
	\item Absenden des zweiten Formulares
	\item weitere Schritte...
\end{enumerate}

In den folgenden Kapiteln werden die einzelnen Use Cases genauer definiert und analysiert, bla blubb und so.

\subsection{Mögliche Aktionen und Effekte}
\begin{enumerate}
	\item 
	\begin{itemize}
		\item Auswahl !Registrieren! in der linken Navigations-Leiste 
	\end{itemize}
	\item 
	\begin{itemize}
		\item Eingabe Persönlicher Daten
		\item Eingabe der Adresse
		\item Eingabe der Heimatadresse
		\item Eingabe Berufliche Einbindung
		\item Auswahl Alumni Angebot
		\item Auswahl zur kostenpflichtigen Mitgliedschaft
		\item Auswahl !Wie haben Sie von der Alumni Würzburg erfahren?!
		\item Markierung zur Einverständniserklärung der Datenschutzerklärung
		\item Markierung zur Einverständniserklärung der Nutzung Personenbezogener Daten
	\end{itemize}
	\item 
	\begin{itemize}
		\item Bestätigung durch den Button !Weiter!
		\item Abbruch durch den Button !Abbruch! 
	\end{itemize}
	\item 
	\begin{itemize}
		\item Eigabe der Daten des Zweiten Formulars...
	\end{itemize}
	\item 
	\begin{itemize}
		\item Absenden des zweiten Formulars...
	\end{itemize}
	\item 
	\begin{itemize}
		\item weitere Schritte...
	\end{itemize}
\end{enumerate}