\section{Evaluationsergebnisse}\label{sec:eval}
Die im Rahmen des Cognitive Walkthrough gesammelten positiven Beobachtungen und detektierten Probleme bezüglich Design und Bedienbarkeit werden im Folgenden aufgelistet. 
Dabei sind pro Problem jeweils die genaue Problembeschreibung, eine Kategorisierung des Problems sowie ein Verbesserungsvorschlag genannt.

Tabelle \ref{tbl:categories} listet die in diesem Zusammenhang verwendeten Kategorien auf. Die Unterteilung der gefundenen Probleme erfolgt anhand der Aufgabenbereiche. Sie sind in der Reihenfolge angeordnet, in der sie in unserem Use-Case aufgetreten sind.

\begin{table}[h]
	\centering\begin{tabular}{|c|l|}
		\hline
		\textbf{Kategorie} & \textbf{Bedeutung} \\
		\hline
		0 & kein Problem \\
		1 & leichtes Problem \\
		2 & mittelschweres Problem \\
		3 & schweres Problem \\
		4 & fatales Problem \\
		\hline
	\end{tabular}
	\caption{Problemschweregrade\label{tbl:categories}}
\end{table}