\section{Evaluationsergebnisse}
Die im Rahmen des Cognitive Walkthrough gemachten positiven Beobachtungen und detektierten Probleme bezüglich Design und Bedienbarkeit werden in den Abschnitten aufgelistet. Dabei werden pro Problem jeweils die genaue Problembeschreibung, eine Kategorisierung des Problems sowie ein Verbesserungsvorschlag genannt.

Tabelle \ref{tbl:categories} listet die in diesem Zusammenhang verwendeten Kategorien auf. Die Unterteilung der gefundenen Probleme erfolgt anhand der Aufgabenbereiche und in der Reihenfolge in denen sie in unserem Use-Case aufgetreten sind.

\begin{table}[h]
	\centering\begin{tabular}{|c|l|}
		\hline
		\textbf{Kategorie} & \textbf{Bedeutung} \\
		\hline
		0 & kein Problem \\
		1 & unauffälliges Problem \\
		2 & leichtes Problem \\
		3 & auffälliges Problem \\
		4 & schwerwiegendes Problem \\
		\hline
	\end{tabular}
	\caption{Problemschweregrade\label{tbl:categories}}
\end{table}